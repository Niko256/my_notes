\chapter{Элементы теории групп}

\begin{flushright}
\small
\itshape
«Математика — это искусство называть разные вещи одним и тем же именем.» \
\upshape
— Анри Пуанкаре
\end{flushright}

\begin{shdef}
    \begin{definition} [Группа]
    \leavevmode \nl

    \textbf{Группой} называется множество \( G \), на котором задана бинарная операция \( \circ : G \times G \to G \), удовлетворяющая следующим аксиомам:
\begin{enumerate}
    \item \textbf{Ассоциативность}: \(\forall a, b, c \in G \) выполняется
    \[
    (a \circ b) \circ c = a \circ (b \circ c).
    \]
    \item \textbf{Существование нейтрального элемента}: Существует элемент \( e \in G \), \\называемый \textbf{нейтральным элементом} (или единицей), такой, что \(\forall a \in G \) выполняется
    \[
    e \circ a = a \circ e = a.
    \]
    \item \textbf{Существование обратного элемента}: \(\forall a \in G \) существует элемент \( a^{-1} \in G \), называемый \textbf{обратным элементом} к \( a \), такой, что
    \[
    a \circ a^{-1} = a^{-1} \circ a = e.
    \]
    \end{enumerate}
    \end{definition}
\end{shdef}

\begin{shdef}
    \begin{definition} [Абелева группа]
    \leavevmode \nl

    Группа \( G \) называется \textbf{абелевой} (или коммутативной), если для любых \( a, b \in G \) выполняется
    \[
    a \circ b = b \circ a.
    \]
    
    \textbf{Примеры:}
    \begin{itemize}
        \item Группа целых чисел по сложению \( (\mathbb{Z}, +) \).
        \item Группа вещественных чисел по сложению \( (\mathbb{R}, +) \).
        \item Группа ненулевых вещественных чисел по умножению \( (\mathbb{R} \setminus \{0\}, \cdot) \).
    \end{itemize}
    \end{definition}
\end{shdef}

\clearpage
\begin{shex} 

Примеры:

    1. \textbf{Группа целых чисел по сложению}:
   \begin{itemize}
       \item Множество: \( \mathbb{Z}\)
       \item Операция: сложение \( + \)
       \item Нейтральный элемент: \( 0 \)
       \item Обратный элемент: для \( a \in \mathbb{Z} \) обратный элемент — это \( -a \)
       \item Свойства: $a + b = b + a \quad \text{(абелева группа)}$
   \end{itemize}

    2. \textbf{Группа биекций множества}:
   \begin{itemize}
       \item Множество: \( \text{Bij}(X) \) — множество всех биекций \(\phi: G \to G \) множества \( X \) на себя.
       \item Операция: композиция отображений \( \circ \): \(\phi \circ \psi(x) = \phi(\psi(x))\)
       \item Нейтральный элемент: тождественное отображение \( \mathcal{I}(x)  = x \quad \forall x \in X\)
       \item Обратный элемент: обратное отображение \( \phi^{-1} \)
       \item Свойства: $ \phi \circ \psi \neq \psi \circ \phi \quad \text{в общем случае (неабелева группа, если \( |X| \geq 3 \))}$
   \end{itemize}

    3. \textbf{Группа невырожденных матриц}:
   \begin{itemize}
       \item Множество: \( GL(n, \mathbb{R}) \) — множество всех квадратных матриц размера \( n \times n \) с ненулевым определителем.
       \item Операция: матричное умножение \( \cdot \)
       \item Нейтральный элемент: единичная матрица \( I \)
       \item Обратный элемент: для матрицы \( A \in GL(n, \mathbb{R}) \) обратный элемент — это \\обратная матрица \( A^{-1} \)
       \item Свойства: $A \cdot B \neq B \cdot A \quad \text{в общем случае (неабелева группа при \( n \geq 2 \))}$
   \end{itemize}
   
4. \textbf{Группа движений \( \text{Isom}(\mathbb{R}^n) \)}:
    \begin{itemize}
       \item Множество: \( \text{Isom}(\mathbb{R}^n) \) — множество всех изометрий (движений) пространства \( \mathbb{R}^n \), то есть биекций \( \phi: \mathbb{R}^n \to \mathbb{R}^n \), сохраняющих расстояние:
       \[
       \|x_1 - x_2\| = \|\phi(x_1) - \phi(x_2)\| \quad \forall x_1, x_2 \in \mathbb{R}^n.
       \]
       \item Операция: композиция отображений \( \circ \): \( \phi \circ \psi(x) = \phi(\psi(x)) \).
       \item Нейтральный элемент: тождественное отображение \( \mathcal{I}_{\mathbb{R}^n} \), заданное как \( \mathcal{I}_{\mathbb{R}^n}(x) = x \) \(\quad \forall x \in \mathbb{R}^n \).
       \item Обратный элемент: для изометрии \( \phi \in \text{Isom}(\mathbb{R}^n) \) обратный элемент — это обратное отображение \( \phi^{-1} \), которое также является изометрией.
    \end{itemize}
\end{shex}

\clearpage


\begin{shdef}
    \begin{definition} [Подгруппа]
    \leavevmode \nl

    Пусть \( G \) — группа с операцией \( \circ \). Подмножество \( H \subset G \) называется \textbf{подгруппой} группы \( G \), если:
    \begin{enumerate}
        \item \( H \) замкнуто относительно операции \( \circ \): \(\forall a, b \in H \) выполняется \( a \circ b \in H \).
        \item \( H \) содержит нейтральный элемент \( e \) группы \( G \): \( e \in H \).
        \item \( H \) замкнуто относительно взятия обратного элемента: \(\forall a \in H \) выполняется \( a^{-1} \in H \).
    \end{enumerate}
    \end{definition}
\end{shdef}

\section{Свойства группы}

\begin{shex}
    \begin{enumerate}
        \item \textbf{Единичный элемент единственен:} 
              \[
              \exists!\, e \in G \quad \text{такой, что} \quad \forall a \in G \quad a \circ e = e \circ a = a.
              \]
              
              \[
              \text{Пусть } e_1, e_2 \text{ — нейтральные элементы. Тогда } e_1 = e_1 \circ e_2 = e_2.
              \]

        \item \textbf{Обратный элемент единственен:} 
              \[
              \forall a \in G \quad \exists!\, a^{-1} \in G \quad \text{такой, что} \quad a \circ a^{-1} = a^{-1} \circ a = e.
              \]
              
              \[
              \text{Пусть } b, c \text{ — обратные к } a. \text{ Тогда } b = b \circ e = b \circ (a \circ c) = (b \circ a) \circ c = e \circ c = c.
              \]

        \item \textbf{Уравнения \( a \circ x = c \) и \( x \circ b = c \) имеют единственное решение:} 
              \[
              \forall a, b, c \in G \quad \exists!\, x \in G \quad \text{такой, что} \quad a \circ x = c \quad \text{и} \quad x \circ b = c.
              \]
              
              \[
              \text{Для } a \circ x = c: \ x = a^{-1} \circ c. \quad \text{Для } x \circ b = c: \ x = c \circ b^{-1}.
              \]

        \item \textbf{Свойство сокращения:} 
              \[
              \forall a, b, c \in G \quad a \circ b = a \circ c \iff b = c.
              \]
              
              \[
              \text{Если } a \circ b = a \circ c, \text{ то } b = c \text{ (умножаем слева на } a^{-1}). 
              \]
              \[
              \text{Если } b \circ a = c \circ a, \text{ то } b = c \text{ (умножаем справа на } a^{-1}).
              \]

        \item \textbf{Обратный элемент композиции:} 
              \[
              \forall a, b \in G \quad (a \circ b)^{-1} = b^{-1} \circ a^{-1}.
              \]
              
              \[
              (a \circ b) \circ (b^{-1} \circ a^{-1}) = a \circ (b \circ b^{-1}) \circ a^{-1} = a \circ e \circ a^{-1} = e.
              \]
              
              \[
              \text{Аналогично, } (b^{-1} \circ a^{-1}) \circ (a \circ b) = e. \text{ Значит, } (a \circ b)^{-1} = b^{-1} \circ a^{-1}.
              \]
              
    \end{enumerate}
\end{shex}

\begin{shdef}
    \begin{definition} [Изоморфизм групп] \( (G, \circ) \) и \( (H, \ast) \) - две группы.
    \leavevmode \nl

    Отображение \( \phi: G \to H \) называется \textbf{изоморфизмом групп}, если:
    \begin{enumerate}
        \item \( \phi \) является \textbf{биекцией} (взаимно однозначным отображением).
        \item \( \phi \) сохраняет групповую операцию, то есть для любых \( a, b \in G \) выполняется:
        \[
        \phi(a \circ b) = \phi(a) \ast \phi(b).
        \]
    \end{enumerate}

    Если такой изоморфизм существует, группы \( G \) и \( H \) называются \textbf{изоморфными}, и обозначается это как \( G \cong H \).
    
    \textbf{Смысл изоморфизма:}
    \begin{itemize}
        \item Изоморфизм групп показывает, что две группы имеют одинаковую структуру, даже если их элементы и операции выглядят по-разному.
        \item Изоморфизм позволяет изучать группы абстрактно, не завися от конкретного представления их элементов и операций.
    \end{itemize}
    \end{definition}
\end{shdef}

\begin{shex}
\textbf{Свойства изоморфизма:}
\leavevmode \nl

\begin{enumerate}
    \item \textbf{Нейтральный элемент переходит в нейтральный:} 
          \[
            \text{Если } e_G \text{ — нейтральный элемент группы } G, \text{ то } \phi(e_G) = e_H.
          \]

    \item \textbf{Обратный элемент переходит в обратный:} 
          \[
          \text{Если } a \in G \text{ и } a^{-1} \text{ — обратный к } a, \text{ то } \phi(a^{-1}) = \phi(a)^{-1}.
          \]
\end{enumerate}
\end{shex}


\boxed{\textbf{1. Нейтральный элемент переходит в нейтральный:}}

Пусть \( \phi: G \to H \) — изоморфизм. Тогда для любого \( a \in G \):
\[
\phi(a) = \phi(a \circ e_G) = \phi(a) \circ \phi(e_G).
\]
Умножая обе части на \( \phi(a)^{-1} \), получаем:
\[
\phi(a)^{-1} \circ \phi(a) \circ \phi(e_G) = \phi(a)^{-1} \circ \phi(a).
\]
Упрощая, получаем:
\[
e_H \circ \phi(e_G) = e_H \quad \lra \quad \phi(e_G) = e_H.
\]

\boxed{\textbf{2. Обратный элемент переходит в обратный:}}

Рассмотрим \( \phi(a \circ a^{-1}) = \phi(e_G) = e_H \).

С другой стороны, \( \phi(a \circ a^{-1}) = \phi(a) \circ \phi(a^{-1}) \).

Таким образом:
\[
\phi(a) \circ \phi(a^{-1}) = e_H, \text{ откуда } \phi(a^{-1}) = \phi(a)^{-1}.
\]

\section{Циклические подгруппы}

\begin{shdef}
    \begin{definition} [Циклическая подгруппа]
    \leavevmode \nl

    Пусть \( G \) — группа, и \( a \in G \) — произвольный элемент. \textbf{Циклической подгруппой}, порождённой элементом \( a \), называется множество всех степеней элемента \( a \):
    \[
    \langle a \rangle = \{ a^n \mid n \in \mathbb{Z} \},
    \]
    где:
    \begin{itemize}
        \item \( a^0 = e \) (нейтральный элемент группы),
        \item \( a^n = \underbrace{a \circ a \circ \dots \circ a}_{n \text{ раз}} \) для \( n > 0 \),
        \item \( a^{-n} = \underbrace{a^{-1} \circ a^{-1} \circ \dots \circ a^{-1}}_{n \text{ раз}} \) для \( n > 0 \).
    \end{itemize}

    Циклическая подгруппа \( \langle a \rangle \) является наименьшей подгруппой группы \( G \), \\содержащей \textbf{образующий} элемент \( a \).
    \end{definition}
\end{shdef}

\begin{shdef}
    \begin{definition} [Бесконечная и конечная циклические подгруппы]
    \leavevmode \nl

    Циклическая подгруппа \( \langle a \rangle = \{ a^n \mid n \in \mathbb{Z} \} \) может быть:
    \begin{itemize}
        \item \textbf{Бесконечной циклической подгруппой}, если все элементы \( a^n \) различны при различных \( n \), то есть:
        \[
        a^m \neq a^s \quad \forall m \neq s.
        \]

        \item \textbf{Конечной циклической подгруппой}, если существуют целые числа \( m \neq s \) такие, что:
        \[
        a^m = a^s.
        \]
        В этом случае, если \( m > s \), то \( a^{m-s} = e \), где \( e \) — нейтральный элемент группы.
    \end{itemize}

    \textbf{Порядок элемента} \( a \) — это наименьшее положительное целое число \( k \), такое что:
    \[
    a^k = e.
    \]
    Если такого \( k \) не существует, то порядок элемента \( a \) считается бесконечным.
    \end{definition}
\end{shdef}


\section*{Пример циклической подгруппы: }

Рассмотрим группу корней \( n \)-й степени из единицы: \(\langle \sqrt[n]{1} \rangle \)
\[
G = \{ z \in \mathbb{C} \mid z^n = 1 \}.
\]
Эти корни имеют вид:
\[
z_k = e^{i \frac{2\pi k}{n}}, \quad k = 0, 1, 2, \dots, n-1.
\]
Группа \( G \) является циклической и порождается элементом:
\[
a = e^{i \frac{2\pi}{n}}.
\]
Циклическая подгруппа, порождённая элементом \( a \), имеет вид:
\[
\langle a \rangle = \{ a^k \mid k = 0, 1, 2, \dots, n-1 \} = \{ e^{i \frac{2\pi k}{n}} \mid k = 0, 1, 2, \dots, n-1 \}.
\]
Элементы подгруппы:
\[
\langle a \rangle = \left\{ 1, e^{i \frac{2\pi}{n}}, e^{i \frac{4\pi}{n}}, \dots, e^{i \frac{2\pi (n-1)}{n}} \right\}.
\]

Рассмотрим случай \( n = 6 \). Корни 6-й степени из единицы:
\[
z_k = e^{i \frac{2\pi k}{6}}, \quad k = 0, 1, 2, 3, 4, 5.
\]

Элементы подгруппы:
\[
\langle a \rangle = \left\{ 1, e^{i \frac{\pi}{3}}, e^{i \frac{2\pi}{3}}, -1, e^{i \frac{4\pi}{3}}, e^{i \frac{5\pi}{3}} \right\}.
\]



\begin{center}
\begin{tikzpicture}[scale=2]

    % Окружность
    \draw (0,0) circle (1);

    % Корни 6-й степени из единицы
    \foreach \k in {0,1,2,3,4,5} {
        \pgfmathsetmacro{\angle}{60*\k}
        \pgfmathsetmacro{\x}{cos(\angle)}
        \pgfmathsetmacro{\y}{sin(\angle)}
        \filldraw (\x,\y) circle (1pt); % Точки без подписей
    }

    % Соединение точек в шестиугольник
    \foreach \k in {0,1,2,3,4,5} {
        \pgfmathsetmacro{\angleA}{60*\k}
        \pgfmathsetmacro{\angleB}{60*(\k+1)}
        \pgfmathsetmacro{\xA}{cos(\angleA)}
        \pgfmathsetmacro{\yA}{sin(\angleA)}
        \pgfmathsetmacro{\xB}{cos(\angleB)}
        \pgfmathsetmacro{\yB}{sin(\angleB)}
        \draw (\xA,\yA) -- (\xB,\yB);
    }

    % Оси
    \draw[->] (-1.5,0) -- (1.5,0) node[right] {Re};
    \draw[->] (0,-1.5) -- (0,1.5) node[above] {Im};

\end{tikzpicture}
\end{center}

\begin{shth}
    \begin{theorem}
    \leavevmode \nl 
    
        Пусть \( G = \langle a \rangle \) — конечная циклическая группа порядка \( k \), и \( e \) — нейтральный элемент группы. Тогда для любого целого числа \( n \) выполняется:
        \[
        a^n = e \iff n \text{ кратен } k.
        \]
        Иными словами, \( a^n = e \) тогда и только тогда, когда \( n = kd \) для некоторого целого числа \( d \).
    \end{theorem}
\end{shth}

\begin{proof}
\leavevmode \nl

\boxed{\Longleftarrow} \quad Пусть \( n = kd \), где \( d \) — целое число. Тогда:
\[
a^n = a^{kd} = (a^k)^d = e^d = e.
\]

\boxed{\Longrightarrow} \quad Пусть \( a^n = e \). Предположим, что \( n \) не кратно \( k \). Тогда, разделив \( n \) на \( k \) с остатком, получим:
\[
n = kd + r, \quad \text{где } 0 < r < k.
\]
Подставим это в равенство \( a^n = e \):
\[
a^n = a^{kd + r} = a^{kd} \cdot a^r = (a^k)^d \cdot a^r = e^d \cdot a^r = a^r.
\]
Таким образом, \( a^r = e \). Но \( 0 < r < k \), что противоречит определению порядка \( k \) как наименьшего положительного числа, для которого \( a^k = e \). Следовательно, наше предположение неверно, и \( n \) должно быть кратно \( k \).
\end{proof}
\vspace{0.3cm}

\begin{shth}
    \begin{theorem}
        Пусть \( G = \langle a \rangle \) — циклическая подгруппа порядка \( k \), где \\\( a \) — образующий элемент, а \( e \) — нейтральный элемент группы. Тогда множество
        \[
        \{ e, a, a^2, \dots, a^{k-1} \}
        \]
        полностью исчерпывает все элементы подгруппы \( G \). Иными словами, все элементы \( G \) имеют вид \( a^n \), где \( n = 0, 1, 2, \dots, k-1 \), и все эти элементы различны.
    \end{theorem}
\end{shth}

\begin{proof}

Докажем, что все элементы подгруппы различны. \\Предположим, что существуют два совпадающих элемента: \( a^m = a^s \), \\где \( 0 \leq m, s < k \) и \( m > s \). Тогда:
\[
a^m = a^s \implies a^{m-s} = e.
\]
Поскольку \( 0 < m - s < k \), это противоречит тому, что \( k \) — наименьшее положительное целое число, для которого \( a^k = e \). Следовательно, все элементы \( e, a, a^2, \dots, a^{k-1} \) различны.

Теперь докажем, что других элементов в подгруппе нет.Рассмотрим произвольный элемент \( a^n \), где \( n \) — целое число. Разделим \( n \) на \( k \) с остатком:
\[
n = kd + r, \quad \text{где } 0 \leq r < k.
\]
Тогда:
\[
a^n = a^{kd + r} = (a^k)^d \cdot a^r = e^d \cdot a^r = a^r.
\]
Таким образом, любой элемент \( a^n \) совпадает с одним из элементов \( e, a, a^2, \dots, a^{k-1} \).
\end{proof}

\begin{shth}
    \begin{theorem}
        Пусть \(G = \langle a \rangle \) — бесконечная циклическая группа, порождённая \\элементом \( a \). Тогда:
        \begin{enumerate}
            \item Элемент \( a^{-1} \) также является образующим группы \( G \), то есть \( \langle a^{-1} \rangle \).
            \item Других образующих в группе \( G \) нет.
        \end{enumerate}
    \end{theorem}
\end{shth}


\begin{proof}
\leavevmode \nl

Для любого целого числа \( n \) выполняется:
\[
a^n = (a^{-1})^{-n}.
\]
Таким образом, любой элемент \( a^n \) может быть выражен через степень \( a^{-1} \). Следовательно, \( a^{-1} \) также является образующим группы \( G \), то есть \( G = \langle a^{-1} \rangle \).

Предположим, что существует другой образующий элемент \( a^s \), где \( s \neq \pm 1 \). Тогда \( a \) должен быть степенью \( a^s \), то есть:
\[
a = (a^s)^l = a^{sl},
\]
где \( l \) — целое число. Поскольку группа \( G \) бесконечна, все степени элемента \( a \) различны. Следовательно:
\[
a^{sl} = a \implies sl = 1.
\]
Так как \( s \) и \( l \) — целые числа, это возможно только если \( s = 1 \) или \( s = -1 \). Таким образом, единственными образующими группы \( G \) являются \( a \) и \( a^{-1} \).
\end{proof}
\vspace{0.2cm}

\begin{shth}
    \begin{theorem} Пусть \( G = \langle a \rangle \) — конечная циклическая группа порядка \( k \). 
    
    Тогда элемент \( a^s \) является образующим группы \( G \) (то есть \( \langle a^s \rangle = G \)) тогда и только тогда, когда числа \( s \) и \( k \) взаимно просты, то есть \( \text{НОД}(s, k) = 1 \).
    \end{theorem}
\end{shth}

\begin{proof}
\leavevmode \nl


\boxed{\Longrightarrow} Пусть \( a^s \) — образующий элемент группы \( G \). 

Предположим, что \( \text{НОД}(s, k) = d > 1 \). Тогда:
\[
k = k_1 \cdot d, \quad s = s_1 \cdot d, \quad \text{где } k_1, s_1 \in \mathbb{Z}.
\]
Рассмотрим степень \( (a^s)^{k_1} \):
\[
(a^s)^{k_1} = (a^{s_1 d})^{k_1} = (a^{k_1 d})^{s_1} = a^{k s_1} = e^{s_1} = e.
\]
Таким образом, \( (a^s)^{k_1} = e \), где \( 0 < k_1 < k \). Это противоречит тому, что порядок \( a^s \) равен \( k \), так как \( k_1 < k \). Следовательно, \( \text{НОД}(s, k) = 1 \).
\\

\boxed{\Longleftarrow} Пусть \( \text{НОД}(s, k) = 1 \). Тогда существуют целые числа \( u \) и \( v \) такие, что:
\[
k u + s v = 1.
\]
Рассмотрим элемент \( a \):
\[
a = a^{k u + s v} = a^{k u} \cdot a^{s v} = (a^k)^u \cdot (a^s)^v = e^u \cdot (a^s)^v = (a^s)^v.
\]
Таким образом, \( a \) является степенью \( a^s \), а значит, \( a^s \) порождает всю группу \( G \), то есть \( \langle a^s \rangle = G \).
\end{proof}