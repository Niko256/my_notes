\section{Приведение квадратичной формы ортогонального \\ оператора к нормальному виду.}

Пусть $f(x,x) = f(x,y)$ - полярная билинейная форма.

Тогда по теореме Рисса (2) : $f(x,y) = \inner{x}{\phi(y)}, \; \phi \in \mathcal{L}(\mathbb{E},\mathbb{E})$

\begin{shth}
    \begin{theorem}
        \leavevmode \nl 
        
        Если форма $f$ - симметрична, то $\phi = \phi^*$
    
    \end{theorem}
\end{shth}

\begin{proof}
    \leavevmode \nl 
    
    $f(x,y) = \inner{x}{\phi(y)},  \; f(y,x) = \inner{y}{\phi(x)} = \inner{\phi(x)}{y} = \inner{x}{\phi^*(y)}$ \\
    
    
    $ \forall x : \inner{x}{\phi(y)} = \inner{x}{\phi^*(y)} \; \lra \phi(y) = \phi^*(y) \; \lra \forall y \quad \phi = \phi^*$
\end{proof}

\begin{shth}
    \begin{theorem}
        $ f(x,y)$ - билинейная форма, $ \mathcal{E}$ - ОНБ. $ \lra A_{f}^{\mathcal{E}} = A_{\phi}^{\mathcal{E}}$.
    \end{theorem}
\end{shth}

\vspace{0.3cm}
\begin{proof}
    
    $ A_{f}^{\mathcal{E}} = (a_{ij}^*) ; \;  A_{\phi}^{\mathcal{E}} = (\hat{a_{ij}})$ \\
    
    $ a_{ij}^* = f(e_{i}, e_{j}) = \inner{e_{i}}{\phi(e_{j})} = \inner{e_{i}}{\sum\limits_{k=1}^{n} \hat{a_{kj}} e_{k}} = \sum\limits_{k=1}^{n} \hat{a_{kj}} \inner{e_{i}}{e_{k}} = \hat{a_{ij}}$
\end{proof}

\vspace{0.3cm}
\begin{shth}
    \begin{theorem}
    \leavevmode \nl 
    
    $f(x,x)$ --- квадратичная форма в $\mathbb{E}$. \quad Тогда $\exists$ ОНБ $\hat{\mathcal{E}}$, в котором:
    \[
    A_{f}^{\hat{\mathcal{E}}} = \Lambda = \begin{pmatrix}
        \lambda_1 & 0 & \cdots & 0 \\
        0 & \lambda_2 & \cdots & 0 \\
        \vdots & \vdots & \ddots & \vdots \\
        0 & 0 & \cdots & \lambda_n
    \end{pmatrix}
    \]
    \[
    \lra f(x,x) = \lambda_1 \hat{e}_1^2 + \lambda_2 \hat{e}_2^2 + \cdots + \lambda_n \hat{e}_n^2
    \]
    \end{theorem}
\end{shth}

\begin{proof}

    $ f(x,x) \sim f(x,y) \sim \phi, \quad \phi = \phi^*$ \\
    
    $ \lra \text{ по спектральной теореме для } \phi \; \exists \text{ ОНБ из собcственных векторов оператора } \phi$. 
    
    $\hat{\mathcal{E}} = \{\hat{e_{1}}, \ldots, \hat{e_{n}} \}, \quad A_{\phi}^{\hat{\mathcal{E}}} = \Lambda \; \lra f(x,x) = \lambda_{1} \hat{\xi_{1}}^2 + \ldots + \lambda_{n} \hat{\xi_{n}}^2$
\end{proof}


\begin{shth}
    \begin{theorem} (Об одновременном приведении пары форм)
        \leavevmode \nl 
        
        пусть $ V $ - Линейное пространство, на котором задана пара квадратичных форм $ f(x,x), g(x,x) $, при этом одна из них положительно определена : $ f(x,x) > 0 $ \\
        
        Тогда в пространстве $V \; \exists \mathcal{E} = \{ e_{1}, \ldots, e_{n}\} : f(x,x) = \xi_{1}^2 + \ldots + \xi_{n}^2, \; g(x,x) = \lambda_{1} \xi_{1}^2 + \lsots + \lambda_{n} \xi_{n}^2$
    \end{theorem}
\end{shth}

\begin{proof} 
    
    $ f^*(x,y) $ полярна к $ f(x,x) $ и задаёт скалярное произведение \\ $ V \to E $. \\
    
    Тогда по теореме 18 $ \exists $ ОНБ $ \mathcal{E} = \{e_{1}, \ldots, e_{n}\}$, где $ g $ имеет каконический вид \\
    $ g(x,x) = \lambda_{1} \xi_{1}^2 + \ldots + \lambda_{n} \xi_{n}^2, \quad f(x,x) = \xi_{1}^2 + \ldots + \xi_{n}^2$
\end{proof}

\vspace{0.4cm}
\begin{shth}
    \begin{theorem} (о нахождении канонического вида для формы $ g $)
    \leavevmode \nl 
    
    пусть заданы формы $ f(x,x) > 0  ; g(x,x); \quad \mathcal{E}$ - исходный базис \\
    
    $ \mathcal{E}' = \{e_{1}', \ldots, e_{n}'\} $; \; $ f(x,x) = (\xi_{1}')^2 + \ldots + (\xi_{n}')^2, \quad g(x,x) = \lambda_{1} (\xi_{1}')^2 + \ldots + \lambda_{n} (\xi_{n}')^2.$ \\
    
    $ \lra \lambda $ - корни уравнения $ |A_{g}^{\mathcal{E}} - \lambda A_{f}^{\mathcal{E}}| = 0. \quad \lra  (A_{g}^{\mathcal{E}} - \lambda_{i} A_{f}^{\mathcal{E}}) \smash{\underset{\downarrow} {x}} = \smash{\underset{\downarrow} {0}}$
        
    \end{theorem}
\end{shth}

\begin{proof}
    
    $ \mathcal{E}' = \{e_{1}', \ldots, e_{n}'\}, \quad A_{f}^{\mathcal{E}'} = E$ \\
    
    $ A_{g}^{\mathcal{E}} = \begin{pmatrix}
        \lambda_1 & 0 & \cdots & 0 \\
        0 & \lambda_2 & \cdots & 0 \\
        \vdots & \vdots & \ddots & \vdots \\
        0 & 0 & \cdots & \lambda_n
    \end{pmatrix}, |A_{g}^{\mathcal{E}'} - \lambda E| = 0 \iff |A_{g}^{\mathcal{E}'} - \lambda A_{f}^{\mathcal{E}'}| = 0$ \\ 
    
    $ A_{f}^{\mathcal{E}'} = T^T A_{f}^{\mathcal{E}} \; T; \; A_{g}^{\mathcal{E}'} = T^T A_{g}^{\mathcal{E}} \; T$ \\ 
    
    $ \lra |T^T A_{g}^{\mathcal{E} T} - \lambda T^T A_{f}^{\mathcal{E}} \; T| = 0 \; \lra |T^T (A_{g}^{\mathcal{E}} - \lambda A_{f}^{\mathcal{E}}) \; T| = 0 $ \\
    
    $\lra |T^T| \; (A_{g}^{\mathcal{E}} - \lambda A_{f}^{\mathcal{E}}) \; |T| = 0 \quad (|T| \neq 0) \; \lra |A_{g}^{\mathcal{E}} - \lambda A_{f}^{\mathcal{E}}| = 0 \quad (|T^T| \neq 0)$ \\
    
    $ e_{i}' = \smash{\underset{\downarrow} {X}} \text{ в базисе } \mathcal{E}, \smash{\underset{\downarrow} {X}}' \text{ в базисе } \mathcal{E}'$ \\ 
    
    $ \smash{\underset{\downarrow} {X}} = T \smash{\underset{\downarrow} {X}}' : (A_{g}^{\mathcal{E}'} - \lambda_{i} E}) \; \smash{\underset{\downarrow} {X}}' = 0$ \\ 
    
    $ (A_{g}^{\mathcal{E}'} - \lambda_{i} A_{f}^{\mathcal{E}'}) \; \smash{\underset{\downarrow} {X}}' = 0 \iff (T^T A_{g}^{\mathcal{E}} \; T - \lambda_{i} T^T A_{f}^{\mathcal{E}} \; T) \; \smash{\underset{\downarrow} {X}}' = T^T (A_{g}^{\mathcal{E}} - \lambda_{i} A_{f}^{\mathcal{E}}) \; T \smash{\underset{\downarrow} {X}}' = 0$
\end{proof}

\chapter{Жорданова форма матрицы}

\section{Многочлены для матриц линейных операторов}

Пусть $ \mathbb{V} = \R, \quad \phi \in \mathcal{L}(\mathbb{V},\mathbb{V})$


\begin{shdef}
\begin{definition}
\leavevmode \nl 

    Пусть $P(t)$ — это многочлен вида:
    \[
    P(t) = a_{0} + a_{1}t + \ldots + a_{n}t^n
    \]
    Тогда $P(\phi)$ определяется как:
    \[
    P(\phi) = a_{0}I + a_{1}\phi + a_{2}\phi^2 + \ldots + a_{n}\phi^n
    \]

    Многочлен $P(t)$ называется аннулирующим для оператора $\phi$, если $P(\phi) = 0$.
\end{definition}

\end{shdef}

\begin{shdef}
    \begin{definition}
    \leavevmode \nl 
    
    Многочлен \( P(a) \) от матрицы \( A \) определяется как:
    \[
    P(a) = a_{0} E + a_{1} A + a_{2} A^2 + \ldots + a_{n} A^n
    \]
    где \( a_0, a_1, \ldots, a_n \) — коэффициенты многочлена, а \( E \) — единичная матрица.
    
    
    \(A \in M_{n}, \quad P(t)\) - многочлен, аннулирующий матрицу $ A$, если \(P(A) = 0\)
\end{definition}
\end{shdef}

\clearpage
\begin{shdef}
    \begin{definition}
    \leavevmode \nl 
    
        \textbf{Минимальный многочлен для матрицы \(A_0\):} \(M_{A}(t)\) \\
        \begin{itemize}
            \item \(M_{A}(A) = 0\)
            \item Старший коэффициент при $ t^k $ = 1 \; $ \lra M_{A}(t) = t^k + \ldots $ 
            \item \(M_{A}(t)\) имеет минимальную возможную степень
        \end{itemize}
        
        \textbf{Минимальный многочлен для оператора \(\phi\):} \(M_{\phi}(t)\) (аналогичные свойства)
    \end{definition}
\end{shdef}

\vspace{0.1cm}
\begin{shth}
\begin{theorem}
\leavevmode \nl

    Пусть \(P(t)\) — аннулирующий многочлен для матрицы \(A\), а \(M_{A}(t)\) — минимальный многочлен. Тогда:
    \[
    P(t) = q(t) \cdot M_{A}(t)
    \]
    для некоторого многочлена \(q(t)\).
\end{theorem}
\end{shth}

\begin{proof}

    Докажем от противного. Предположим, что
    \[
    P(t) = q(t) \cdot M_{A}(t) + r(t), \text{где }\deg(r(t)) < \deg(M_{A}(t))
    \]
    

    Поскольку \(P(t)\) аннулирует \(A\), имеем:
    \[
    0 = P(A) = q(A) \cdot M_{A}(A) + r(A).
    \]
    Так как \(M_{A}(A) = 0 \; \lra 0 = r(A) \)
    
    Однако, это противоречит тому, что \(\deg(r(t)) < \deg(M_{A}(t))\), поскольку \(M_{A}(t)\) — минимальный многочлен. Следовательно, \(r(t) = 0\), и
    \[
    P(t) = q(t) \cdot M_{A}(t).
    \]
\end{proof}


\begin{shth}
\begin{theorem}[Гамильтона-Кэли]
\leavevmode \nl 

    Пусть \( A \in M \), и \( X_{A}(t) \) - характеристический многочлен матрицы \( A \), т.е. \\ \( X_{A}(t) = \det(A - t \cdot E) \). Тогда \( X_{A}(t) \) является аннулирующим многочленом для \( A \), т.е. \( X_{A}(A) = 0 \).
\end{theorem}
\end{shth}

\clearpage
\begin{proof}

    Пусть \( A \in M \) и \( A - t \cdot E = (c_{ij})(t) \). Обозначим \( B(t) = (b_{ij}(t)) \), где \( b_{ij}(t) \) - алгебраические дополнения элементов \( C_{ij} \).

    Тогда \( b_{ij}(t) = b_{ij}^{(0)} + b_{ij}^{(1)} t + \ldots + b_{ij}^{(n-1)} t^{n-1} \).

    Следовательно, \( B(t) = B^{(0)} + B^{(1)} t + \ldots + B^{(n-1)} t^{n-1} \), и \((A - t \cdot E)^{-1} = \frac{1}{\det(A - t \cdot E)} \cdot B(t) \).

    Поскольку \( B(t) = (A - t \cdot E)^{-1} \cdot X_{A}(t) \) для \( t \neq \lambda_{i} \), где \( \lambda_{i} \) - корни характеристического уравнения, умножим на \( (A - t \cdot E) \) справа:

    \[
    B(t)(A - t \cdot E) = X_{A}(t) E
    \]

    Это приводит к равенству многочленов:

    \[
    (B^{(0)} + B^{(1)} t + \ldots + B^{(n-1)} t^{n-1})(A - t \cdot E) = (\alpha_{0} + \alpha_{1} t + \ldots + \alpha_{n} t^n) E
    \]

    Убирая ограничение \( t \neq \lambda_{i} \), получаем систему уравнений:

    \[
    \begin{cases}
        B^{(0)} A = \alpha_{0} E \\
        B^{(1)} A - B^{(0)} E = \alpha_{1} E \\
        B^{(2)} A - B^{(1)} E = \alpha_{2} E \\
        \vdots \\
        B^{(n)} A - B^{(n-1)} E = \alpha_{n} E \\
    \end{cases}
    \]

    Таким образом, \( 0 = \alpha_{0} E + \alpha_{1} A + \ldots + \alpha_{n} A^n = X_{A}(A) \).

    Следовательно, характеристический многочлен делится на минимальный, и все корни минимального многочлена вещественны.
\end{proof}

\section{Корневые подпространства}

Пусть \( \mathbb{V} \) - Линейное пространство, \(\phi \in \mathcal{L}(\mathbb{V}, \mathbb{V}), \quad  \) все корни \(X_{\phi}(t)\) вещественные.

\begin{shdef}
    \begin{definition}
        \leavevmode \\
        
        \(x_{0} \neq 0\) - Корневой вектор, соответствующий \(\lambda_{0}\), если \(\exists k \in \mathbb{N} : (\phi - \lambda_{0} I)^k x_{0} = 0\).
        
        \(k\) - высота корневого вектора \(x_{0}\), если \((\phi - \lambda_{0} I)^k x_{0} = 0\), но \((\phi - \lambda_{0} I)^{k-1} x_{0} \neq 0\).
        
        \(\Longrightarrow\) собственный вектор - частный случай корневого вектора с высотой 1.
    \end{definition}
\end{shdef}

\begin{shth}
\begin{theorem}[О корневом подпространстве]
    \leavevmode \nl 
    
    Пусть \(\lambda\) - собственное число оператора \(\phi\), тогда 
        $$\mathcal{L}_{(\alpha)} = \{ \text{все корневые вектора } \sim \alpha \oplus \smash{\underset{\downarrow} {0}} \}.$$
    
    \(\lra \mathcal{L}_{(\alpha)}\) является подпространством \(\mathbb{V}\) и называется корневым подпространством.
\end{theorem}
\end{shth}

\begin{proof}
\leavevmode \nl 

    Поскольку \(\smash{\underset{\downarrow} {0}} \in \mathcal{L}_{(\alpha)}\), то \(\mathcal{L}_{(\alpha)} \neq \varnothing\). Пусть \(x_{1}, x_{2} \in \mathcal{L}_{(\alpha)}\) и \(\xi, \mu \in \mathbb{R}\).

    Рассмотрим \(y = \xi x_{1} + \mu x_{2}\). Тогда \(y \in \mathcal{L}_{(\alpha)}\).

    \(\Longrightarrow \exists k, s \in \mathbb{N} : (\phi - \alpha I)^k x_{1} = 0\) и \((\phi - \alpha I)^s x_{2} = 0\).

    Пусть \(m = \max(k, s)\). Тогда:
    \[
    (\phi - \alpha I)^m y = (\phi - \alpha I)^m (\xi x_{1} + \mu x_{2}) = \xi (\phi - \alpha I)^m x_{1} + \mu (\phi - \alpha I)^m x_{2} = 0.
    \]

    Следовательно, \(\mathcal{L}_{(\alpha)}\) является подпространством. Пусть \(\mathbb{V}_{1} \subseteq \mathbb{V}\) и \(\phi \in \mathcal{L}(\mathbb{V}, \mathbb{V})\).

    Тогда \(\mathbb{V}_{1}\) инвариантно относительно \(\phi\), и \(\mathcal{E}_{1} = \{ e_{1}, \ldots, e_{k} \}\) можно дополнить до базиса \(\mathcal{E}\).
\end{proof}


\vspace{0.3cm}
\begin{shth}
    \begin{theorem} [О высоте корневого вектора]
        \leavevmode \nl 
        
        \(dim(\mathcal{L}_{\alpha}) = m, \; x \neq \smash{\underset{\downarrow} {0}} \in \mathcal{L}_{\alpha}, \; k \) - $ \max $ высота  корневого вектора \(x.\; \lra k \leq m\)
    \end{theorem}
\end{shth}


\begin{proof}
    Рассмотрим ограничение оператора \( \phi \) на корневое \\ подпространство \( \mathcal{L}_{\alpha} \), обозначим его как \( \phi_1 = \phi|_{\mathcal{L}_{\alpha}} \). 

    Поскольку \( \mathcal{L}_{\alpha} \) — корневое подпространство, характеристический многочлен \( \phi_1 \) имеет вид:
    \[
    \phi_1(t) = (t - \alpha)^m.
    \]

    Это означает, что минимальный многочлен \( X_{\phi_1}(\lambda) \) оператора \( \phi_1 \) также имеет вид:
    \[
    X_{\phi_1}(\lambda) = (\lambda - \alpha)^m.
    \]

    Таким образом, \(\forall x \in \mathcal{L}_{\alpha} \) выполняется:
    \[
    (\phi_1 - \alpha I)^m(x) = 0.
    \]

    Пусть \( x \neq 0 \) — корневой вектор максимальной высоты \( k \). Это означает, что:
    \[
    (\phi_1 - \alpha I)^k(x) = 0,
    \]
    \(\forall j < k \) выполняется:
    \[
    (\phi_1 - \alpha I)^j(x) \neq 0.
    \]

    Поскольку \( X_{\phi_1}(\lambda) = (\lambda - \alpha)^m \) и \( m = \dim(\mathcal{L}_{\alpha}) \), максимальная высота корневого вектора \( x \) не может превышать \( m \). Следовательно, \( k \leq m \). 
    
    Таким образом, доказательство показывает, что высота корневого вектора \\ ограничена размерностью корневого подпространства.
\end{proof}

