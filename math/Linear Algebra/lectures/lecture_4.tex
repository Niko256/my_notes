\section{Нормальные операторы в $E$}

\begin{shstmt}
    \begin{statement}
        \leavevmode \nl 
        
        $ \text{Оператор $\phi$ - нормальный $ \lra \; \phi^*$ - нормальный } $
    \end{statement}
\end{shstmt}

\begin{proof}
    \leavevmode \nl 
    
    $$\phi^* = \psi \lra \phi \cdot \psi = \psi \cdot \phi \lra (\phi \psi)^* = (\psi \phi)^*$$
    
    $$\lra \psi^* \cdot \phi^* = \phi^* \cdot \psi^* \lra \psi^* \psi = \psi \psi^*$$
    
    $$\lra \psi = \phi^* - \text{Нормальный}$$
\end{proof}

\textbf{Свойства нормальных операторов:}
\vspace{0.3cm}

\begin{enumerate}
     
    \item $\inner{\phi(x)}{\phi(y)} = \inner{\phi(x)^*}{\phi(y)^*}$ 

    \item $\|\phi(x)\| = \|\phi^*(x)\|$ 
    
    \item $(\phi - \lambda I)$ — нормальный оператор.
    
    \item Пусть $x$ — собственный вектор для $\phi$, соответствующий собственному числу $\lambda$, тогда $x$ — собственный вектор для $\phi^*$, соответствующий $\lambda$.
    
    \item Собственные вектора, соответствующие различным собственным числам нормального оператора, взаимно ортогональны.
    
    \item Пусть $e$ — собственный вектор для $\phi$, тогда $E = L(e) \oplus L(e)^{\bot}$, а подпространства $L(e)$ и $L(e)^{\bot}$ инвариантны относительно $\phi$ и $\phi^*$.
    
    \item Если $\mathbb{E}$ — ортонормированный базис (ОНБ), то $A_{\phi}^{\mathbb{E}}$ — нормальная матрица.
    
    \item Если $\mathbb{E}$ — ОНБ, то $A_{\phi}^{\mathbb{E}}$ соответствует нормальному оператору $\phi$ .
\end{enumerate}

\begin{proof}
    \leavevmode \nl
    \begin{enumerate}
        \item 
        \[
        \inner{\phi(x)}{\phi(y)} = \inner{x}{\phi(\phi^*(y))} = \inner{\phi^*(y)}{\phi^*(x)} = \inner{\phi^*(x)}{\phi^*(y)}
        \]

        \item 
        \[
        \| \phi(x)\| = \sqrt{\inner{\phi(x)}{\phi(x)}} = \sqrt{\inner{\phi^*(x)}{\phi^*(x)}} = \|\phi^*(x)\|.
        \]

        \item 
        \[
        (\phi - \lambda I)(\phi - \lambda I)^* = (\phi - \lambda I)(\phi^* - \lambda I) = \phi \phi^* - \lambda \phi^* - \lambda \phi + \lambda^2 I.
        \]
        \[
        (\phi - \lambda I)^*(\phi - \lambda I) = (\phi^* - \lambda I)(\phi - \lambda I) = \phi^* \phi - \lambda \phi - \lambda \phi^* + \lambda^2 I.
        \]
        \[
        \text{Так как } \phi \phi^* = \phi^* \phi, \text{ то } (\phi - \lambda I) \text{ — нормальный оператор.}
        \]

        \item 
        \[
        \text{Пусть } x \text{ — собственный вектор для } \phi, \text{ соответствующий } \lambda. \text{ Тогда:}
        \]
        \[
        \phi(x) = \lambda x \implies (\phi - \lambda I)(x) = 0.
        \]
        \[
        \|(\phi - \lambda I)(x)\| = 0 = \|(\phi - \lambda I)^*(x)\| \implies (\phi - \lambda I)^*(x) = 0 \implies \phi^*(x) = \lambda x.
        \]

        \item 
        \[
        \text{Пусть } e_1 \text{ и } e_2 \text{ — собственные вектора для } \phi, \text{ соответствующие } \lambda_1 \text{ и } \lambda_2 \text{ соответственно. Тогда:}
        \]
        \[
        \inner{\phi(e_1)}{e_2} = \inner{\lambda_1 e_1}{e_2} = \lambda_1 \inner{e_1}{e_2}.
        \]
        \[
        \inner{e_1}{\phi^*(e_2)} = \inner{e_1}{\lambda_2 e_2} = \lambda_2 \inner{e_1}{e_2}.
        \]
        \[
        \text{Так как } \phi \text{ — нормальный, то } \lambda_1 \inner{e_1}{e_2} = \lambda_2 \inner{e_1}{e_2} \implies (\lambda_1 - \lambda_2) \inner{e_1}{e_2} = 0.
        \]
        \[
        \text{Если } \lambda_1 \neq \lambda_2, \text{ то } \inner{e_1}{e_2} = 0 \implies e_1 \perp e_2.
        \]

        \item 
        \[
        \text{Пусть } e \text{ — собственный вектор для } \phi. \text{ Тогда:}
        \]
        \[
        E = L(e) \oplus L(e)^{\bot}.
        \]
        \[
        \phi(e) = \lambda e \in L(e) \implies L(e) \text{ инвариантно относительно } \phi.
        \]
        \[
        \phi^*(e) = \lambda e \in L(e) \implies L(e) \text{ инвариантно относительно } \phi^*.
        \]
        \[
        \text{Следовательно, } L(e)^{\bot} \text{ инвариантно относительно } \phi \text{ и } \phi^*.
        \]

        \item 
        \[
        \text{Пусть } \mathbb{E} \text{ — ОНБ. Тогда:}
        \]
        \[
        A_{\phi^*}^{\mathbb{E}} = (A_{\phi}^{\mathbb{E}})^T.
        \]
        \[
        \text{Так как } \phi \text{ — нормальный, то:}
        \]
        \[
        A_{\phi}^{\mathbb{E}} (A_{\phi}^{\mathbb{E}})^T = (A_{\phi}^{\mathbb{E}})^T A_{\phi}^{\mathbb{E}}.
        \]
        \[
        \text{Следовательно, } A_{\phi}^{\mathbb{E}} \text{ — нормальная матрица.}
        \]

        \item 
        \[
        \text{Пусть } \mathbb{E} \text{ — ОНБ. Тогда:}
        \]
        \[
        A_{\phi}^{\mathbb{E}} (A_{\phi}^{\mathbb{E}})^T = (A_{\phi}^{\mathbb{E}})^T A_{\phi}^{\mathbb{E}}.
        \]
        \[
        \text{Это эквивалентно } \phi \phi^* = \phi^* \phi \implies \phi \text{ — нормальный оператор.}
        \]
    \end{enumerate}
\end{proof}
\clearpage
\section{Основные свойства самосопряжённого оператора}

\begin{shstmt}
    \begin{statement}
        \leavevmode \nl 
        
        \begin{enumerate}
            \item 
            \[
            \phi \phi^* = \phi^* \phi = \phi \phi \lra \phi - \text{ Нормальный оператор} \nl 
            
            \text{(для самосопряжённых операторов верны 8 свойств нормальных операторов)}
            \]
            \item
            \[
            \text{Пусть $\mathbb{E} $ - ОНБ и } \phi = \phi^* \lra A_{\phi}^{\mathbb{E}} = (A_{\phi}^{\mathbb{E}})^T \lra A_{\phi}^{\mathbb{E}} - \text{симметрична}
            \]

            \item 
            \[
            \text{$\phi$ - самосопряжённый $\lra$ все корни характеристического многочлена $\R$}
            \]
        \end{enumerate}
    \end{statement}
\end{shstmt}


\begin{proof}
    (третьего свойства)
    \leavevmode \nl 
    
    \[
    \text{От противного. } \lambda_{0} - \text{корень} \lra \lambda_{0} = \alpha_{0} + i\beta_{0}, \; \beta_{0} \neq 0
    
    \lra (A_{\phi}^{\mathbb{E}} - \lambda_{0} E)  \smash{\underset{\downarrow}{\xi}} = 0, \;  \quad \xi_{0} = x_{0} + i w_{0} - \text{ нетривиальное решение}. \nl 
    
    \lra (A_{\phi}^{\mathbb{E}} - (\alpha_{0} + i \beta_{0})E)( \smash{\underset{\downarrow}{x_{0}}} + i  \smash{\underset{\downarrow}{w_{0}}}) =  \smash{\underset{\downarrow}{0}} = 
    \begin{cases}
        A_{\phi}^{\mathbb{E}}  \smash{\underset{\downarrow}{x_{0}}} = \alpha_{0}  \smash{\underset{\downarrow}{x_{0}}} - \beta_{0} w_{0} \\
        A_{\phi}^{\mathbb{E}}  \smash{\underset{\downarrow}{w_{0}}} = \alpha_{0}  \smash{\underset{\downarrow}{w_{0}}} + \beta_{0}  \smash{\underset{\downarrow}{x_{0}}}
    \end{cases}
    \]
    
    \[
    \text{Пусть $u_{0}$ =  $\smash{\underset{\downarrow}{x_{0}}}$ в базисе $\mathcal{E}$, \; $v_{0} = \smash{\underset{\downarrow}{w_{0}}}$ в базисе $\mathcal{E}$} 
    
    \[
\begin{cases}
    \phi(u_{0}) = \smash{\underset{\downarrow}{x_{0}}} \text{ в } \mathcal{E} \\
    \phi(v_{0}) = \smash{\underset{\downarrow}{w_{0}}} \text{ в } \mathcal{E} \\
\end{cases}
\]

\[
\implies \inner{\phi(u_{0})}{v_{0}} = \alpha_{0} \inner{u_{0}}{v_{0}} - \beta_{0} ||v_{0}||^2 = \inner{u_{0}}{\phi(v_{0})} = \alpha_{0} \inner{u_{0}}{v_{0}} + \beta_{0} ||u_{0}||^2 \implies \beta_{0} (||v_{0}||^2 + ||u_{0}||^2) = 0
\]

\[
\implies \beta_{0} = 0 \text{ или } (||v_{0}||^2 + ||u_{0}||^2) = 0
\]

\[
\text{Поскольку } \beta_{0} \neq 0, \text{ это приводит к противоречию, так как } ||v_{0}||^2 + ||u_{0}||^2 \neq 0 \text{ для } \xi_{0} \neq 0.
\]

\[
\text{Следовательно, все корни характеристического многочлена должны быть действительными.}
\]
\end{proof}

\section{Спектральная теория для самосопряжённых операторов}

\begin{shth}
    \begin{theorem}
    \leavevmode \nl 
    
        $\phi \in \mathcal{L}(E, E), \; \phi = \phi^*. \quad \text{Тогда в $E \; \exists $ ОНБ из собственных векторов оператора $\phi$}$
    \end{theorem}
\end{shth}


\begin{proof} По индукции по размерности.
    
    \begin{enumerate}
        \item $ \boxed{\text{База индукции}} \quad e_{1} \neq 0 - \text{ базис в } E. \quad \|e_{1}\| = 1, \quad \phi(e_{1}) = \lambda e_{1}$
        \item $ \boxed{\text{Шаг индукции}} - \text{ верно, тогда для } n: \nl  

        \lambda_{1} \in \R, \; e_{1} \sim \lambda_{1}, \; \|e_{1}\| = 1 \quad \lra E = \mathcal{L}(e_{1}) \oplus \mathcal{L}^{\bot}(e_{1}). \nl

         \text{так как } \phi = \phi^* \; \lra \phi - \text{нормальный} \lra \mathcal{L}(e_{1}) \text{ и } \mathcal{L}^{\bot}(e_{1}) \text{ инвариантны относительно } \phi \nl 
         
         \text{Пусть } E_{1} = \mathcal{L}^{\bot}(e_{1}) \lra \dim(E_{1}) = n - 1, \; \phi_{1} = \phi = \phi^*_{1} \in \mathcal{L}(E_{1}, E_{1}) \\ 
         
         (\text{рассматриваем $\phi_{1}$ как ограничение на $\phi$ на подпространство} E_{1})
         
        \lra \text{по предположению индукции } \exists \text{ ОНБ } \mathcal{E} = \{ e_{2}, \cdots e_{n}\} \text{ из собственных векторов } \phi_{1} \\ 

        \lra \text{ так как } e_{1} \bot \{e_{2}, \cdots, e_{n}\} \; \lra \mathcal{E} = \{e_{1}, \csots, e_{n}\} - \text{ искомый базис}$
        
    \end{enumerate}
\end{proof}

\begin{shstmt}
    \begin{statement}
        \leavevmode \\
        
        $ A = A^T - \text{ симметричная матрица. Тогда } \exists T - \text{ ортогональная матрица:} \\

 T^T A T = \Lambda = \begin{bmatrix} \lambda_1 & 0 & \cdots & 0 \\ 0 & \lambda_2 & \cdots & 0 \\ \vdots & \vdots & \ddots & \vdots \\ 0 & 0 & \cdots & \lambda_n \end{bmatrix}, \text{ причём } \lambda_{1}, \cdots, \lambda_{n} - \text{ с.ч. матрицы } A$
    \end{statement}
\end{shstmt}

\begin{proof}
    Пусть $\mathcal{E}$ — ортонормированный базис (ОНБ), $\phi \in \mathcal{L}(E, E)$ и $A_{\phi}^{\mathcal{E}} = A$.
    
    \[
    A_{\phi}^{\mathcal{E}} = (A_{\phi}^{\mathcal{E}})^T \iff \phi = \phi^* \iff \exists \mathcal{E}' \text{ — ОНБ из собственных векторов для } \phi: \overline{\mathcal{E}'} = \overline{\mathcal{E}} \cdot T.
    \]
    
    Было доказано, что $T$ — ортогональная матрица перехода от базиса $\mathcal{E}$ к базису $\mathcal{E}'$:
    \[
    T^{-1} = T^T \iff A_{\phi}^{\mathcal{E}'} = T^{-1} \cdot A_{\phi}^{\mathcal{E}} \cdot T = T^T \cdot A_{\phi}^{\mathcal{E}} \cdot T.
    \]
    
    Так как $\phi$ — самосопряженный оператор, его матрица в базисе $\mathcal{E}'$ диагональна с собственными значениями $\lambda_1, \lambda_2, \dots, \lambda_n$ на диагонали:
    \[
    \Lambda = \begin{bmatrix} 
        \lambda_1 & 0 & \cdots & 0 \\ 
        0 & \lambda_2 & \cdots & 0 \\ 
        \vdots & \vdots & \ddots & \vdots \\ 
        0 & 0 & \cdots & \lambda_n 
    \end{bmatrix}.
    \]
\end{proof}


\section{Ортогональные операторы}

\begin{shdef}
    \begin{definition}
        Оператор $ \phi $ ортогонален, если $ \phi^* \phi = \phi \phi^* = I$
    \end{definition}
\end{shdef}

\vspace{0.2cm}


\textbf{Свойства ортогональных операторов:}
\begin{enumerate}
    \item $\langle \phi(x), \phi(y) \rangle = \langle x, y \rangle$
    \item $\|\phi(x)\| = \|x\|$
    \item $\mathcal{E} = \{e_{1}, \ldots, e_{n}\}$ — ОНБ
    \item Если оператор ортогональный, то его собственные числа равны $\lambda = \pm 1$ \\
          (если пространство унитарное, то $|\lambda| = 1$)
    \item $\mathcal{E}$ — ОНБ $\iff A_{\phi}^{\mathcal{E}}$ ортогональная матрица
    \item Пусть $A$ — ортогональная матрица $\lra \det(A) = \pm 1$
    \item $\mathcal{E}' = \mathcal{E} T$, где $\mathcal{E}$ и $\mathcal{E}'$ — ОНБ $\iff T$ — ортогональная
    \item Если $\mathcal{E}$ — ОНБ и $T$ — ортогональная, то $\mathcal{E}' = \mathcal{E} T$ — ОНБ
\end{enumerate}

\begin{proof}
    \leavevmode
    \begin{enumerate}
        \item 
            \[
            \langle \phi(x), \phi(y) \rangle = \langle x, \phi^* \phi(y) \rangle = \langle x, I y \rangle = \langle x, y \rangle
            \]
        \item
            \[
            \|\phi(x)\|^2 = \langle \phi(x), \phi(x) \rangle = \langle x, x \rangle = \|x\|^2
            \]
        \item 
            \[
            \|\phi(e_{i})\| = \|e_{i}\| = 1, \quad \langle \phi(e_{i}), \phi(e_{j}) \rangle = \langle e_{i}, e_{j} \rangle = 0, \text{ если } i \neq j \Rightarrow \{\phi(e_{1}), \ldots, \phi(e_{n})\} \text{ — ОНБ}
            \]
        \item 
        \[
        \lambda \text{ — собственное число, соответствующее собственному вектору } e \Rightarrow \phi(e) = \lambda e
        \]
        \[
        \|\phi(e)\| = \|\lambda e\| = |\lambda| \|e\| \Rightarrow |\lambda| = 1 \Rightarrow \lambda = \pm 1
        \]
        \item 
        \[
        \boxed{\Rightarrow} \quad \mathcal{E} \text{ — ОНБ}, \; \phi^* \phi = \phi \phi^* = I \Rightarrow (A_{\phi}^{\mathcal{E}})^T \cdot A_{\phi}^{\mathcal{E}} = A_{\phi}^{\mathcal{E}} \cdot (A_{\phi}^{\mathcal{E}})^T = E, \Rightarrow A_{\phi}^{\mathcal{E}} \text{ — ортогональный}
        \]
        \[
        \boxed{\Leftarrow} \quad \phi \sim A_{\phi}^{\mathcal{E}}, \; \phi^* \sim (A_{\phi}^{\mathcal{E}})^T \Rightarrow A_{\phi}^{\mathcal{E}} \cdot (A_{\phi}^{\mathcal{E}})^T = (A_{\phi}^{\mathcal{E}})^T \cdot A_{\phi}^{\mathcal{E}} = E \Rightarrow \phi \phi^* = \phi^* \phi = I
        \]
        \item 
        \[
        A \text{ — ортогональный} \Rightarrow A A^T = E, \quad \det(A A^T) = \det(A)^2 = \det(E) = 1
        \]
        \item Было доказано ранее
        \item 
        \[
        \mathcal{E} = \{e_{1}, \ldots, e_{n}\}, \; \mathcal{E}' = \{e_{1}',\ldots, e_{n}'\}
        \]
        \[
        e_{i}' = \sum\limits_{k=1}^{n} t_{ki} e_{k}, \; e_{j}' = \sum\limits_{s=1}^{n} t_{sj} e_{s}
        \]
        \[
        \langle e_{i}', e_{j}' \rangle = \langle \sum\limits_{k=1}^{n} t_{ki} e_{k}, \sum\limits_{s=1}^{n} t_{sj} e_{s} \rangle = \sum_{k=1}^{n} \sum\limits_{s=1}^{n} t_{ki} t_{sj} \langle e_{k}, e_{s} \rangle = \sum\limits_{k=1}^{n} t_{ki} t_{kj} = \delta_{ij} \Rightarrow \mathcal{E}' \text{ — ОНБ}
        \]
    \end{enumerate}
\end{proof}

\begin{shlem}
    \begin{lemma} (О клеточной диагональной матрице)
    \leavevmode \nl 
    
 Пусть \( V \) — линейное пространство, \( V = V_{1} \oplus V_{2} \), \; \( \phi \in \mathcal{L}(V, V) \), и \( \mathcal{E} = \mathcal{E}_{1} \cup \mathcal{E}_{2} \) — базис в \( V \), где \( \mathcal{E}_{1} \) — базис в \( V_{1} \), а \( \mathcal{E}_{2} \) — базис в \( V_{2} \). \nl 
 Тогда матрица оператора \( \phi \) в базисе \( \mathcal{E} \) имеет клеточно-диагональный вид:

\[
A_{\phi}^{\mathcal{E}} = \begin{bmatrix}
A_{\phi|_{V_{1}}}^{\mathcal{E}_{1}} & 0 \\
0 & A_{\phi|_{V_{2}}}^{\mathcal{E}_{2}}
\end{bmatrix}
\]
    \end{lemma}
\end{shlem}


\begin{proof}
\leavevmode \nl 

Пусть \(\dim(V_{1}) = k\) и \(\dim(V_{2}) = n-k\). Базисы \(\mathcal{E}_{1}\) и \(\mathcal{E}_{2}\) определены как:
\[
\mathcal{E}_{1} = \{ e_1, \ldots, e_k \}, \quad \mathcal{E}_{2} = \{ e_{k+1}, \ldots, e_n \}
\]

Рассмотрим действие оператора \(\phi\) на элементы базиса:

\[
\begin{cases}
\phi(e_1) = \alpha_{11} e_1 + \ldots + \alpha_{k1} e_k + 0 e_{k+1} + \ldots + 0 e_n \\
\vdots \\
\phi(e_k) = \alpha_{1k} e_1 + \ldots + \alpha_{kk} e_k + 0 e_{k+1} + \ldots + 0 e_n \\
\phi(e_{k+1}) = 0 e_1 + \ldots + 0 e_k + \alpha_{k+1,k+1} e_{k+1} + \ldots + \alpha_{nk+1} e_n \\
\vdots \\
\phi(e_n) = 0 e_1 + \ldots + 0 e_k + \alpha_{k+1,n} e_{k+1} + \ldots + \alpha_{nn} e_n
\end{cases}
\]

Таким образом, матрица оператора \(\phi\) в базисе \(\mathcal{E}\) имеет клеточно-диагональный вид:

\[
A_{\phi}^{\mathcal{E}} = \begin{bmatrix}
\alpha_{11} & \cdots & \alpha_{1k} & 0 & \cdots & 0 \\
\vdots & \ddots & \vdots & \vdots & \ddots & \vdots \\
\alpha_{k1} & \cdots & \alpha_{kk} & 0 & \cdots & 0 \\
0 & \cdots & 0 & \alpha_{k+1,k+1} & \cdots & \alpha_{nk+1} \\
\vdots & \ddots & \vdots & \vdots & \ddots & \vdots \\
0 & \cdots & 0 & \alpha_{k+1,n} & \cdots & \alpha_{nn}
\end{bmatrix} = 
A_{\phi}^{\mathcal{E}} = \begin{bmatrix}
A_{\phi|_{V_{1}}}^{\mathcal{E}_{1}} & 0 \\
0 & A_{\phi|_{V_{2}}}^{\mathcal{E}_{2}}
\end{bmatrix}
\]
\end{proof}

\begin{shth}
    \begin{theorem} (О существовании канонического базиса для орт. оператора)
    \leavevmode \nl 
    
        Пусть \(\phi \in \mathcal{L}(E, E)\) — ортогональный оператор. Тогда существует ОНБ \(\mathcal{E} = \{e_1, \ldots, e_n\}\), в котором матрица оператора \(\phi\) имеет вид:
\[
A_{\phi}^{\mathcal{E}} = \begin{bmatrix}
J_1 & 0 & \cdots & 0 \\
0 & J_2 & \cdots & 0 \\
\vdots & \vdots & \ddots & \vdots \\
0 & 0 & \cdots & J_m
\end{bmatrix}
\]
где каждый блок \(J_i\) представляет собой либо:
\begin{itemize}
    \item матрицу размера \(1 \times 1\): \((1)\) или \((-1)\)
    \item матрицу поворота размера \(2 \times 2\) на угол \(\theta\): 
    \(\begin{bmatrix} \cos(\theta) & -\sin(\theta) \\ \sin(\theta) & \cos(\theta) \end{bmatrix}\)
\end{itemize}
    \end{theorem}
\end{shth}

