\begin{shth}
    \begin{theorem} (Об инвариантности корневого подпространства)
        \leavevmode \nl 
            
            Пусть \(\lambda \) - собственное число оператора \(\phi \). \(\lra \mathcal{L}_{\lambda}\) инвариантно относительно \(\phi\)
    \end{theorem}
\end{shth}



\begin{proof}
    Рассмотрим вектор \(x \in \mathcal{L}_{\lambda}\). По определению корневого подпространства, это означает, что существует такое натуральное число \(k\), что:
    \[
    (\phi - \lambda I)^k(x) = 0.
    \]
    Нам нужно показать, что \(\phi(x) \in \mathcal{L}_{\lambda}\), то есть:
    \[
    (\phi - \lambda I)^k(\phi(x)) = 0.
    \]

    Поскольку \((\phi - \lambda I)^k(x) = 0\), мы можем применить оператор \(\phi\) к обеим частям этого уравнения:
    \[
    \phi((\phi - \lambda I)^k(x)) = \phi(0) = 0.
    \]

    Используя ассоциативность операторов, мы можем переписать это как:
    \[
    (\phi - \lambda I)^k(\phi(x)) = 0.
    \]

    Таким образом, \(\phi(x) \in \mathcal{L}_{\lambda}\), что доказывает инвариантность корневого подпространства \(\mathcal{L}_{\lambda}\) относительно оператора \(\phi\).
\end{proof}



\begin{shth}
    \begin{theorem}[О линейной независимости корневых векторов разной высоты]
    \leavevmode \nl 
    
        Пусть $x_{1}, \ldots, x_{m} \in \mathcal{L}_{\lambda}$ и $k_{1}, \ldots, k_{m}$ — их соответствующие высоты. Тогда векторы $x_{1}, \ldots, x_{m}$ линейно независимы.
    \end{theorem}
\end{shth}

\begin{proof}[Доказательство (от противного)]

    Предположим, что векторы $x_{1}, \ldots, x_{m}$ ЛЗ. 
    
    Тогда существует нетривиальная линейная комбинация:
    \[
    \lambda_{1} x_{1} + \ldots + \lambda_{m} x_{m} = 0, \quad (\exists j : \lambda_{j} \neq 0)
    \]
   

    Применим оператор $(\phi - \lambda I)^{k_{m} - 1}$ к этой комбинации:
    \[
    \lambda_{1} (\phi - \lambda I)^{k_{m} - 1} x_{1} + \ldots + \lambda_{m-1} (\phi - \lambda I)^{k_{m} - 1} x_{m-1} + \lambda_{m} (\phi - \lambda I)^{k_{m} - 1} x_{m} = 0.
    \]

    Поскольку $k_{1}, \ldots, k_{m-1} \leq k_{m} - 1$, то:
    \[
    \lambda_{m} (\phi - \lambda I)^{k_{m} - 1} x_{m} = 0 \implies \lambda_{m} = 0.
    \]

    Аналогично, применяя оператор $(\phi - \lambda I)^{k_{m} - 1}$ к оставшимся векторам, получаем, что $\lambda_{m-1} = 0, \ldots, \lambda_{1} = 0$. Это противоречит предположению о нетривиальности линейной комбинации.

    Следовательно, векторы $x_{1}, \ldots, x_{m}$ ЛНЗ.
\end{proof}

\vspace{0.3cm}
\begin{shth}
    \begin{theorem}[О разложении пространства в прямую сумму корневых \\ подпространств]
    \leavevmode \nl 
    
        Пусть $\phi \in \mathcal{L}(\mathbb{V}, \mathbb{V})$ и все корни характеристического многочлена $\phi$ вещественные: $\lambda_{1}, \ldots, \lambda_{s}$. Тогда пространство $\mathbb{V}$ разлагается в прямую сумму корневых \\ подпространств: 
        \[
        \mathbb{V} = \mathcal{L}_{\lambda_{1}} \oplus \ldots \oplus \mathcal{L}_{\lambda_{s}}.
        \]
    \end{theorem}
\end{shth}

\begin{proof}
\leavevmode \nl 

    Рассмотрим произвольный ненулевой вектор $x \in \mathbb{V}$. Пусть $m_{x}(t)$ — минимальный многочлен для $x$, который удовлетворяет следующим условиям:
    \begin{enumerate}
        \item $m_{x}(\phi)(x) = 0$.
        \item $m_{x}(t)$ имеет старший коэффициент 1.
        \item $m_{x}(t)$ минимальной возможной степени.
    \end{enumerate}
    Известно, что $m_{x}(t)$ делит характеристический многочлен $m_{\phi}(t)$, и все его корни вещественные. Следовательно, $m_{x}(t)$ можно представить в виде:
    \[
    m_{x}(t) = (t - \lambda_{1})^{k_1} \cdot \ldots \cdot (t - \lambda_{r})^{k_{r}}.
    \]

    Мы утверждаем, что $x$ можно разложить как $x = x_{1} + \ldots + x_{r}$, где $x_{i} \in \mathcal{L}_{\lambda_{i}}$. 
    
Докажем это утверждение методом математической индукции по $r$.

    \textbf{База индукции:} $r = 1$. В этом случае $m_{x}(t) = (t - \lambda_{1})^{k_{1}}$, и, следовательно, \\$(\phi - \lambda_{1} I)^{k_{1}}(x) = 0$, что означает $x \in \mathcal{L}_{\lambda_{1}}$. Таким образом, $x = x + 0 + \ldots + 0$.

    \textbf{Шаг индукции:} Предположим, что утверждение верно для $r-1$. \\ Рассмотрим случай $r$. Пусть
    \[
    m_{x}(t) = f(t) \cdot g(t),
    \]
    где $f(t) = (t - \lambda_{1})^{k_{1}} \cdot \ldots \cdot (t - \lambda_{r-1})^{k_{r-1}}$ и $g(t) = (t - \lambda_{r})^{k_{r}}$. 
    
    По теореме из школьной алгебры существуют многочлены $u(t)$ и $v(t)$ такие, что
    \[
    u(t) \cdot f(t) + v(t) \cdot g(t) = 1.
    \]

    Применяя оператор $\phi$, получаем:
    \[
    u(\phi) \cdot f(\phi) + v(\phi) \cdot g(\phi) = I.
    \]

    Следовательно,
    \[
    x = u(\phi) \cdot f(\phi)(x) + v(\phi) \cdot g(\phi)(x) = x_{1} + x_{2},
    \]
    где $x_{1} = u(\phi) \cdot f(\phi)(x)$ и $x_{2} = v(\phi) \cdot g(\phi)(x)$.


    Докажем, что $x_{1}$ и $x_{2}$ принадлежат соответствующим корневым \\подпространствам. Поскольку $g(\phi)(x_{1}) = 0$, $m_{x_{1}}(t)$ делит $g(t)$, и по предположению индукции $x_{1}$ можно разложить в сумму векторов из корневых подпространств. 
    
    Аналогично, $f(\phi)(x_{2}) = 0$, и $m_{x_{2}}(t)$ делит $f(t)$, что также позволяет разложить $x_{2}$ в сумму векторов из корневых подпространств.

    Таким образом, любой вектор $x \in \mathbb{V}$ можно разложить в сумму векторов из корневых подпространств, что доказывает, что $\mathbb{V} = \mathcal{L}_{\lambda_{1}} \oplus \ldots \oplus \mathcal{L}_{\lambda_{s}}$.
\end{proof}

\vspace{0.4cm}
\begin{proof} (Что сумма является прямой).
\leavevmode \nl 

    Докажем утверждение по индукции по количеству слагаемых в прямой сумме.

    \textbf{База индукции:} Рассмотрим случай $\mathcal{L}_{\lambda} \oplus \mathcal{L}_{\beta}$. Предположим, что $\mathcal{L}_{\lambda} \cap \mathcal{L}_{\beta} = \{0\}$.

    Пусть $x \neq 0$ и $x \in \mathcal{L}_{\lambda} \cap \mathcal{L}_{\beta}$. Тогда $x \in \mathcal{L}_{\lambda}$, следовательно, $(\phi - \lambda I)^k (x) = 0$. 
    
    Также $x \in \mathcal{L}_{\beta}$, следовательно, $(\phi - \beta I)^s (x) = 0$. Это означает, что $m_{x}(t)$ делит оба многочлена $(t - \lambda)^k$ и $(t - \beta)^s$. Поскольку $\lambda \neq \beta$, $m_{x}(t) = 1$, что противоречит предположению $x \neq 0$. Следовательно, $x = 0$.

    \textbf{Шаг индукции:} Предположим, что для $m$ подпространств $\mathcal{L}_{\lambda_{1}}, \ldots, \mathcal{L}_{\lambda_{m}}$ \\выполнено $\mathcal{L}_{\lambda_{1}} \oplus \ldots \oplus \mathcal{L}_{\lambda_{m}} \cap \mathcal{L}_{\lambda_{m+1}} = \{0\}$.

    Пусть $x_{m+1} \neq 0$, $x_{m+1} \in \mathcal{L}_{\lambda_{m+1}}$, и $x_{m+1} \in \mathcal{L}_{\lambda_{1}} \oplus \ldots \oplus \mathcal{L}_{\lambda_{m}}$. Тогда $x_{m+1} = x_{1} + \ldots + x_{m}$, где $x_i \in \mathcal{L}_{\lambda_i}$.

    Т.к. $(\phi - \lambda_{m+1} I)^{k_{m+1}}(x_{m+1}) = 0$, то $(\phi - \lambda_{m+1} I)^{k_{m+1}}(x_{1}) + \ldots + (\phi - \lambda_{m+1} I)^{k_{m+1}}(x_{m}) = 0$. 
    
    Это означает, что $y_{1} + \ldots + y_{m} = 0$, где $y_i = (\phi - \lambda_{m+1} I)^{k_{m+1}}(x_{i})$.

    Поскольку сумма $\mathcal{L}_{\lambda_{1}} \oplus \ldots \oplus \mathcal{L}_{\lambda_{m}}$ прямая, то $y_{i} = 0$ для всех $i$. Следовательно, $x_{i} \in \mathcal{L}_{\lambda_{i}} \cap \mathcal{L}_{\lambda_{m+1}} = \{0\}$, что означает $x_{i} = 0$ для всех $i$. 
    
    Это противоречит предположению $x_{m+1} \neq 0$.

    Таким образом, $z \in \mathcal{L}_{\lambda_{1}} \oplus \ldots \oplus \mathcal{L}_{\lambda_{m}} \oplus \mathcal{L}_{\lambda_{m+1}}$ представляется единственным образом 
    
    $$z = z_{1} + \ldots + z_{m} + z_{m+1},$$
    
    где $z_i \in \mathcal{L}_{\lambda_i}$. Следовательно, сумма $\mathcal{L}_{\lambda_{1}} \oplus \ldots \oplus \mathcal{L}_{\lambda_{m+1}}$ является прямой.
\end{proof}

\vspace{0.3cm}

\begin{shcor}
    \begin{corollary}[О ЛНЗ векторов из разных корневых подпространств]
        \leavevmode \nl 
        
        Пусть $x_1 \in \mathcal{L}_{\alpha_1}, \ldots, x_m \in \mathcal{L}_{\alpha_m}$ — ненулевые векторы из различных корневых \\подпространств (т.е. $\alpha_i \neq \alpha_j$ при $i \neq j$). Тогда система векторов $\{x_1, \ldots, x_m\}$ линейно независима.
    \end{corollary}
\end{shcor}

\begin{proof}
    От противного:
    
    Предположим, что система векторов $\{x_1, \ldots, x_m\}$ линейно зависима. Тогда \\существуют числа $\lambda_1, \ldots, \lambda_m$, не все равные нулю, такие что:
    \[
        \lambda_1 x_1 + \ldots + \lambda_m x_m = 0
    \]
    \clearpage
    Заметим, что:
    \begin{itemize}
        \item $\lambda_1 x_1 \in \mathcal{L}_{\alpha_1}$ (т.к. корневое подпространство является линейным)
        \item $\vdots$
        \item $\lambda_m x_m \in \mathcal{L}_{\alpha_m}$
    \end{itemize}
    
    По доказанной теореме, сумма корневых подпространств 
    $\mathcal{L}_{\alpha_1} \oplus \ldots \oplus \mathcal{L}_{\alpha_m}$ является прямой. 
    
    Следовательно, нулевой вектор может быть представлен единственным образом как сумма векторов из этих подпространств, а именно:
    \[
        0 = 0 + \ldots + 0
    \]
    
    Таким образом:
    \[
        \begin{cases}
            \lambda_1 x_1 = 0 \\
            \vdots \\
            \lambda_m x_m = 0
        \end{cases}
    \]
    
    Поскольку все $x_i \neq 0$ по условию, получаем:
    \[
        \lambda_1 = \ldots = \lambda_m = 0
    \]
    
    Это противоречит предположению о том, что не все $\lambda_i$ равны нулю. 
    \\Следовательно, система векторов $\{x_1, \ldots, x_m\}$ линейно независима.
\end{proof}

\vspace{0.3cm}

\begin{shdef}
    \begin{definition}
        \leavevmode \nl 
        
        Жордановой клеткой порядка $k$ с собственным числом $\lambda$ называется квадратная матрица размера $k \times k$ вида:
        
        $J_k(\lambda) = \begin{pmatrix}
            \lambda & 1 & 0 & \cdots & 0 \\
            0 & \lambda & 1 & \cdots & 0 \\
            0 & 0 & \lambda & \ddots & 0 \\
            \vdots & \vdots & \ddots & \ddots & 1 \\
            0 & 0 & \cdots & 0 & \lambda
        \end{pmatrix}$
    \end{definition}
\end{shdef}

\begin{shdef}
    \begin{definition}
        \leavevmode \nl 
        
        Жордановой формой матрицы $A$ называется блочно-диагональная матрица вида:
        
        $J = \begin{pmatrix}
            J_{k_1}(\lambda_1) & 0 & \cdots & 0 \\
            0 & J_{k_2}(\lambda_2) & \cdots & 0 \\
            \vdots & \vdots & \ddots & \vdots \\
            0 & 0 & \cdots & J_{k_s}(\lambda_s)
        \end{pmatrix}$
        
        где $J_{k_i}(\lambda_i)$ — жордановы клетки, соответствующие собственным числам $\lambda_i$.
    \end{definition}
\end{shdef}


\begin{shdef}
    \begin{definition}
        \leavevmode \nl 
        
        Пусть $\phi: V \to V$ — линейный оператор и $x \in V$. Циклическим подпространством $\{x\}_{\phi}$ называется наименьшее подпространство, содержащее вектор $x$ и \\инвариантное относительно оператора $\phi$.
        
        Другими словами, если $x$ — корневой вектор, соответствующий собственному \\числу $\lambda_i$, то:
        
        $$\{x\}_{\phi} = \text{span}\{x, \phi(x), \phi^2(x), \dots, \phi^{k-1}(x)\}$$
        
        где $k$ - наименьшее натуральное число, при котором система векторов $x, \phi(x), \phi^2(x), \dots, \phi^k(x)$ становится \\линейно зависимой.
    \end{definition}
\end{shdef}

\vspace{0.3cm}

\begin{shth}
    \begin{theorem}
        Пусть \(\phi: V \to V\) — линейный оператор, и \(x \in V\) — корневой вектор, соответствующий собственному значению \(\lambda\). Тогда циклическое подпространство \(\{x\}_{\phi}\) существует.
    \end{theorem}
\end{shth}

\begin{proof}
    Рассмотрим последовательность векторов:
    \[
    x, \phi(x), \phi^2(x), \dots, \phi^k(x),
    \]
    где $k$ — наименьшее натуральное число, при котором эта система становится линейно зависимой. Такое $k$ существует, так как пространство $V$ конечномерно.

    Построим подпространство:
    \[
    \{x\}_{\phi} = \operatorname{span}\{x, \phi(x), \phi^2(x), \dots, \phi^{k-1}(x)\}.
    \]

    \textbf{1. Содержит $x$:} \\
    По построению, $x \in \{x\}_{\phi}$.

    \textbf{2. Инвариантность относительно $\phi$:} \\
    Для любого вектора $v \in \{x\}_{\phi}$ его образ $\phi(v)$ также принадлежит $\{x\}_{\phi}$. 
    
    Действительно, если
    \[
    v = \alpha_0 x + \alpha_1 \phi(x) + \dots + \alpha_{k-1} \phi^{k-1}(x),
    \]
    то
    \[
    \phi(v) = \alpha_0 \phi(x) + \alpha_1 \phi^2(x) + \dots + \alpha_{k-1} \phi^k(x).
    \]
    
    Поскольку $\phi^k(x)$ линейно выражается через предыдущие векторы (так как система линейно зависима), $\phi(v) \in \{x\}_{\phi}$.

    \textbf{3. Минимальность:} \\
    
    Пусть $W$ — любое подпространство, содержащее $x$ и инвариантное относительно $\phi$. Тогда:
    \begin{itemize}
        \item $x \in W$
        \item $\phi(x) \in W$ (так как $W$ инвариантно)
        \item $\phi^2(x) \in W$
        \item $\vdots$
        \item $\phi^{k-1}(x) \in W$
    \end{itemize}
    Следовательно, $\{x\}_{\phi} \subseteq W$.

    Таким образом, $\{x\}_{\phi}$ — наименьшее подпространство, содержащее $x$ и инвариантное относительно $\phi$.
\end{proof}

\vspace{0.2cm}

\begin{shth}
    \begin{theorem}[О базисе циклического подпространства]
    \leavevmode \nl 
    
    Пусть \( x \) — корневой вектор высоты \( k \sim \) собственному значению \( \lambda \). Тогда векторы
    \[
    \begin{cases}
        e_0 = x, \\
        e_1 = \phi(x), \\
        \quad \vdots \\
        e_{k-1} = \phi^{k-1}(x)
    \end{cases}
    \]
    образуют базис в циклическом подпространстве \( \{x\}_{\phi} \).
    \end{theorem}
\end{shth}


\begin{proof}
    Рассмотрим корневой вектор \( x = e_0 \) высоты \( k \). По определению корневого вектора:
    \[
    (\phi - \lambda I)^k x = 0, \quad \text{но} \quad (\phi - \lambda I)^{k-1} x \neq 0.
    \]
    Определим векторы:
    \[
    \begin{cases}
        e_1 = (\phi - \lambda I) x, \\
        e_2 = (\phi - \lambda I)^2 x, \\
        \quad \vdots \\
        e_{k-1} = (\phi - \lambda I)^{k-1} x.
    \end{cases}
    \]
    Заметим, что:
    \begin{itemize}
        \item \( e_0 = x \) имеет высоту \( k \),
        \item \( e_1 = (\phi - \lambda I) x \) имеет высоту \( k-1 \),
        \item $\vdots$
        \item \( e_{k-1} = (\phi - \lambda I)^{k-1} x \) имеет высоту \( 1 \).
    \end{itemize}
    Векторы \( e_0, e_1, \ldots, e_{k-1} \) линейно независимы, так как они имеют разную высоту.

    Пусть \( y \in \{x\}_{\phi} \). Тогда \( y \) можно выразить как:
    \[
    y = p(\phi)(x),
    \]
    где \( p(t) \) — многочлен степени не выше \( k-1 \):
    \[
    p(t) = \beta_0 + \beta_1 (t - \lambda) + \ldots + \beta_{k-1} (t - \lambda)^{k-1}.
    \]
    Подставляя \( \phi \) вместо \( t \), получаем:
    \[
    y = \beta_0 x + \beta_1 (\phi - \lambda I) x + \ldots + \beta_{k-1} (\phi - \lambda I)^{k-1} x.
    \]
    Это эквивалентно:
    \[
    y = \beta_0 e_0 + \beta_1 e_1 + \ldots + \beta_{k-1} e_{k-1}.
    \]
    Таким образом, мы показали, что векторы \( e_0, e_1, \ldots, e_{k-1} \) линейно независимы и любой вектор \( y \in \{x\}_{\phi} \) можно выразить как их линейную комбинацию:

    Таким образом, \( \{e_0, e_1, \ldots, e_{k-1}\} \) образует базис в циклическом подпространстве \( \{x\}_{\phi} \).

    Рассмотрим двойственный базис \( \epsilon^* = \{e_1^*, e_2^*, \ldots, e_k^*\} \), который строится путём перестановки векторов \( \{e_0, e_1, \ldots, e_{k-1}\} \) в обратном порядке:
    \[
    \begin{cases}
        e_1^* = e_{k-1}, \\
        e_2^* = e_{k-2}, \\
        \quad \vdots \\
        e_k^* = e_0.
    \end{cases}
    \]
    Тогда:
    \[
    \begin{cases}
        (\phi - \lambda I) e_1^* = (\phi - \lambda I) e_{k-1} = 0, \\
        (\phi - \lambda I) e_2^* = (\phi - \lambda I) e_{k-2} = e_{k-1} = e_1^*, \\
        \quad \vdots \\
        (\phi - \lambda I) e_k^* = (\phi - \lambda I) e_0 = e_1 = e_{k-1}^*.
    \end{cases}
    \]
    Это означает, что \( \epsilon^* \) — циклический базис (или жорданова цепочка) в \( \{x\}_{\phi} \).
\end{proof}

\vspace{0.2cm}

\begin{shth}
    \begin{theorem}[О существовании Жордановой формы матрицы]
        \leavevmode \nl 
        
        Пусть \( U \) — инвариантное подпространство, состоящее из корневых векторов, \\соответствующих собственному значению \( \lambda \). \\Тогда \( U \) является прямой суммой циклических подпространств.
    \end{theorem}
\end{shth}

\begin{shth}
    \begin{theorem}
        \leavevmode \nl 
        
        Пусть \( \phi \in \mathcal{L}(\mathbb{V}, \mathbb{V}) \). Если все корни характеристического уравнения действительные, то в пространстве \( \mathbb{V} \) существует базис \( \mathcal{E} \), в котором матрица \( A_{\mathcal{E}}^{\phi} \) имеет Жорданову форму.
    \end{theorem}
\end{shth}

