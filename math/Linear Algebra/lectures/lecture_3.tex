\section{Линейные, билинейные и полуторалинейные формы в Евклидовом пространстве}

Пусть $ f: \mathbb{V} \to E  \quad f(\alpha x + \beta y) = \alpha f(x) + \beta f(y)$

\begin{shlem}
    \begin{lemma} (О скалярном произведении)
    \leavevmode \nl
    
    Пусть $ \mathbb{V} - \text{ это } E \text{ или } U, \quad \forall y \in \mathbb{V} \quad \inner{x_{1}}{y} = \inner{x_{2}}{y} \quad \lra x_{1} = x_{2}$
    \end{lemma}
\end{shlem}

\begin{proof}
    \leavevmode \nl 
    
    $ \forall y \quad \inner{x_{1}}{y} = \inner{x_{2}}{y} \; \lra \; \inner{x_{1} - x_{2}}{y} = 0$
    
    $ \text{ Пусть } y = x_{1} - x_{2} \; \lra \; \inner{x_{1} - x_{2}}{x_{1} - x_{2}} = 0 \; \lra \; x_{1} - x_{2} = 0 \; \lra \; x_{1} = x_{2}$
\end{proof}

\vspace{0.4cm}

\begin{shth}
    \begin{theorem} (Теорема Рисcа о линейном функционале)
        
        $ \text{ Пусть } \mathbb{V} - \text{ это } E, \quad f - \text{ это линейный функционал из } E^* \; \lra \nl 
        \lra \; \exists ! \; h \in E : f(x) = \inner{x}{h}$
    \end{theorem}
\end{shth}

\begin{proof}
    \leavevmode \nl 
    
    $$ \mathcal{E} = \{ e_{1}, \ldots, e_{n} \} - \text{ ОНБ в } E, \quad x = \xi_{1} e_{1} + \ldots + \xi_{n} e_{n}, \quad \forall x \in E$$
    
    
    $$f(x) = \xi_{1} f(e_{1}) + \ldots + \xi_{n} f(e_{n})$$
    
    
    $$ \mu_{1} = f(e_{1}), \; \ldots \;, \mu_{n} = f(e_{n}), \quad h = \mu_{1} e_{1} + \ldots + \mu_{n} e_{n}$$
    
    
    $$ \lra \; \xi_{1} \mu_{1} + \ldots + \xi_{n} \mu_{n} = \inner{x}{h}$$
    
    
    $$ \text{ теперь покажем, что вектор $h$ единственен. Пойдём от противного.} $$
    

    $$ \text{ Пусть } \exists h_{1}, h_{2} \in E \; \lra \; f(x) = \inner{x}{h_{1}}, \; f(x) = \inner{x}{h_{2}} \quad \lra \inner{x}{h_{1}} = \inner{x}{h_{2}} \; \nl \forall x \in E$$
    
    
    $$ \lra \text{ По лемме } h_{1} = h_{2}$$
\end{proof}

\clearpage

\begin{shth}
    \begin{theorem} (Теорема типа Рисса о представлении билинейной формы \nl в евклидовом пространстве)
        
        $ \text{ Пусть $f(x, y)$ - билинейная форма в $E$}$
        
        $\quad \lra \exists ! \; \phi \in \mathcal{L}(E, E) : f(x, y) = \inner{x}{\phi(y)}$
        
    \end{theorem}
\end{shth}

\begin{proof}
\leavevmode \nl 

    \begin{enumerate}
        \item 
        
        $$\text{ фиксируем } y \in E, \; f(x,y) \in E^*$$
        
        $$\text{ По теореме Рисса } \exists ! \; h_{y} : f(x,y) = \inner{x}{h_{y}} $$
        
        $$\phi: E \to E : \; \phi(y) = h_{y}, \; y_{1} \to h_{y_{1}}, \ldots , y_{n} \to h_{y_{n}}$$
        
        $$f(x, y_{1} + y_{2}) = f(x, y_{1}) + f(x, y_{2}) = \inner{x}{\phi(y_{1})} + \inner{x}{\phi(y_{2})}$$
        
        $$\text{ так как } \phi(y_{1} + y_{2}) = \phi(y_{1}) \; +  \; \phi(y_{2}) \; \lra \; f(x, \alpha y) = \alpha f(x, y) = \alpha \inner{x}{h_{y}} = \alpha \inner{x}{\phi(y)}$$
        
        $$\inner{x}{\phi(\alpha y)} \leftrightarrow \inner{x}{\alpha \phi(y)}. \text{ По лемме } \pgu(\alpha y) = \alpha \phi(y)$$

        $$\lra \exists \phi \in \mathcal{L}(E, E) : f(x, y) = f(x, \phi(y))$$
        
        
        \item 

        $$\text{ Докажем единственность } \phi \text{ от противного. Пусть } \quad
\begin{align}
f(x, y) &= \inner{x}{\phi_{1}(y)} \\
f(x, y) &= \inner{x}{\phi_{2}(y)} \end{align}}
 \quad \forall x \in E$$
        
        $$\text{ По лемме } \phi_{1}(y) = \phi_{2}(y) \; \lra \phi_{1} = \phi_{2}$$
        
        \item 
        
        $$\text{ если } \mathbb{V} = U, \text{ то } g(x,y) - \text{ квадратичная форма}$$
        
        $$\lra \exists ! \; \phi \in \mathcal{L}(U, U) : g(x, y) = \inner{x}{\phi(y)} \text{ и } \exists ! \; \psi \in \mathcal{L}(U, U) : \; g(x, y) = \inner{\psi(x)}{y}$$
    \end{enumerate}
\end{proof}


\clearpage

\chapter{Линейные операторы в $E$ и $U$}

\section{Сопряжённые операторы}

$ \text{ Пусть $\mathbb{V}$ - это $E$ или $U$}, \; \phi \in \mathcal{L}(\mathbb{V}, \mathbb{V})$

\begin{shdef}
    \begin{definition}
    \leavevmode \nl 
    
        $\phi^* \text{ называется сопряжённым оператором к $\phi$, если } \forall x, y \in \mathbb{V} \ \ \inner{\phi(x)}{y} = \inner{x}{\phi^* (y)}$
    \end{definition}
\end{shdef}

\begin{shth}
    \begin{theorem}
        $\forall \phi \in \mathcal{L}(\mathbb{V}, \mathbb{V}) \quad \exists ! \; \phi^* : \phi^* \in \mathcal{L}(\mathbb{V}, \mathbb{V}) $
    \end{theorem}
\end{shth}

\begin{proof}
    \leavevmode \nl 
    
    $$\inner{\phi(x)}{y} - \text{ полуторалинейная форма }$$
    
    $$\text{ По теореме Рисса } \inner{\phi(x)}{y} = \inner{x}{\psi(y)}$$

    $$\text{ По определению } \psi = \phi^* \in \mathcal{L}(\mathbb{V}, \mathbb{V}). \text{ Пусть есть 2 сопряжённых оператора } \phi_{1}^*, \phi_{2}^*$$
    
    $$\inner{\phi(x)}{y} = \inner{x}{\phi_{1}(y)^*}, \quad \inner{\phi(x)}{y} = \inner{x}{\phi_{2}(y)^*}$$
    
    $$\lra \inner{x}{\phi_{1}(y)^*} = \inner{x}{\phi_{2}(y)^*} \text{ (По лемме $\phi_{1}^* = \phi_{2}^*$)} $$
\end{proof}

\clearpage

\begin{shex}
    \textbf{Свойства $\phi^*$:}
    \begin{enumerate}
        \item $ I^* = I$
        \item $(\alpha \phi + \mu \psi)^* = 
        \begin{cases}
            \alpha \phi^* + \mu \psi^* & \text{в } E \\
            \overline{\alpha} \phi^* + \overline{\mu} \psi^* & \text{в } U
        \end{cases}$
        \item $ (\phi^*)^* = \phi$
        \item $ (\phi \psi)^* = \psi^* \phi^*$
        \item $\text{ если } \exists \phi^{-1} \in \mathcal{L}(\mathbb{V}, \mathbb{V}), \; \text{ то } \; \exists (\phi^*)^{-1}, \; (\phi^*)^{-1} = (\phi^{-1})^*$
        \item $ \text{ если } \mathcal{E} - \text{ ОНБ, то } A^{\phi}_{\phi^*} = 
        \begin{cases}
            A^{e^T}_{\phi} \text{ в } E \\
            (A^{e}_{\phi})^* \text{ в } U
        \end{cases}$
    \end{enumerate}
\end{shex}

\begin{proof}
    \leavevmode \nl 
    
        \begin{enumerate}
        \item 
        
        $$\inner{I x}{y} = \inner{x}{y} = \inner{x}{I y} \lra I^* = I$$
        \item 
        
        $$\text{ Докажем для } E: \quad $ $ \inner{(\lambda \phi + \mu \psi) (x)}{y} = \inner{\lambda \phi(x) + \mu \psi(x)}{y} = \lambda \inner{\phi(x)}{y} + \mu \inner{\psi(x)}{y} = $$
        
$$ = \lambda \inner{x}{\phi(y)^*} + \mu \inner{x}{\psi(y)^*} = \inner{x}{\lambda \phi(y)^*} + \inner{x}{\mu \psi(y)^*} = \inner{x}{\lambda \phi(y)^* + \mu \psi(y)^*}.$$

$$\lra \; \text{ По лемме } (\lambda \phi + \mu \psi)^* = \lambda \phi^* + \mu \psi^* $$

        \item
        $$\text{ Докажем для } U:$$
        $$\inner{\phi(x)}{y} = \inner{x}{\phi(y)^*} = \overline{\inner{\phi(y)^*}{x}} = \overline{\inner{y}{(\phi(x)^*})^*} = $$
        
        $$= \inner{(\phi(x)^*)^*}{y} \; \lra \; \phi = (\phi^*)^* \text{ по лемме }$$
        
       \item
       $$\inner{\phi \psi(x)}{y} = \inner{\psi(x)}{\phi(y)^*} = \inner{x}{(\phi \psi)^* (y)} = \inner{x}{\psi^* \phi(y)^*}$$
       
       $$\lra \text{ по лемме } (\phi \psi)^* = \psi^* \phi^* \; \forall x \in \mathbb{V}$$
       
       \item
       $$\exists \phi^{-1} : \phi^{-1} \phi = \phi \phi^{-1} = I$$

       $$(\phi^{-1} \phi)^* = (\phi \phi^{-1})^* = I^* = I \quad \lra \phi^* (\phi^{-1})^* = \phi^* \psi = (\phi^{-1})^* \phi^* = \psi \phi^* = I$$

       $$\lra \psi^* \psi = \psi \psi^* = I$$
       
       \item
       Пусть \(\mathcal{E}\) — ОНБ. Обозначим матрицу оператора \(\phi\) в базисе \(\mathcal{E}\) как \(A^{e}_{\phi} = (a_{ij})\), а матрицу оператора \(\phi^*\) в том же базисе как \(A^{e}_{\phi^*} = (b_{ij})\).

\begin{enumerate}
    \item Случай 1: Базис \(\mathcal{E}\) — ОНБ в пространстве \(E\).
    
    В этом случае, матрица \(A^{e}_{\phi^*}\) оператора \(\phi^*\) в базисе \(\mathcal{E}\) совпадает \nl
с транспонированной матрицей \(A^{e}_{\phi}\):
    
    \[
    A^{e}_{\phi^*} = (A^{e}_{\phi})^T
    \]
    
    Это следует из определения сопряженного оператора в \nl 
ортонормированном базисе.
    
    \item Случай 2: Базис \(\mathcal{E}\) — ОНБ в пространстве \(U\).
    
    В этом случае, матрица \(A^{e}_{\phi^*}\) оператора \(\phi^*\) в базисе \(\mathcal{E}\) совпадает с матрицей \((A^{e}_{\phi})^*\):
    
    \[
    A^{e}_{\phi^*} = (A^{e}_{\phi})^*
    \]
    
    Это также следует из определения сопряженного оператора в \nl
ортонормированном базисе, но с учетом комплексного сопряжения и \nl 
транспонирования.
\end{enumerate}

    \end{enumerate}
\end{proof}

\begin{shdef}
    \begin{definition}
        \leavevmode \nl 
        
        $ \mathbb{V}_{1} \subseteq \mathbb{V}, \quad \mathbb{V}_{1} - \text{инвариантно относительно $\phi$, если $ \forall x \in \mathbb{V}_{1} \quad \phi(x) \in \mathbb{V}_{1}$}$
    \end{definition}
\end{shdef}

\vspace{0.3cm}
$ \text{Если $ \mathbb{V}_{1}$ инвариантно относительно $\phi$, то $ \mathbb{V}_{1}$ инвариантно относительно $ \phi^* $}$

$ \lra 0 = \inner{\phi(x)}{y} = \inner{x}{\phi(y)^*} \lra \quad \phi(y)^* \in \mathbb{V}_{1}$

\vspace{0.6cm}

\begin{shth}
    \begin{theorem}
        \leavevmode \nl 
        
        $ \text{Пусть $\mathbb{E}$ - произвольный базис, $\Gamma$ - Матрица Грамма}$
        
        $ \lra A_{\phi^*}^{\mathbb{E}} = \Gamma^{-1} (A_{\phi}^{\mathbb{E}})^T \; \Gamma, \quad \text{если  $\mathbb{E}$ - ОНБ, то \; $A_{\phi^*}^{\mathbb{E}} = (A_{\phi}^{\mathbb{E}})^T$}$
    \end{theorem}
\end{shth}


\begin{proof}
    \leavevmode \nl 
    
    $$ x = \xi_{1}e_{1} + \ldots + \xi_{n}e_{n}, \quad y =  \mu_{1}e_{1} + \ldots + \mu_{n}e_{n}$$
    
    $$\phi(x) = \xi_{1}' e_{1} + \ldots + \xi_{n}' e_{n}, \quad \phi^*(y) = \mu_{1}' e_{1} + \ldots + \mu_{n}' e_{n}$$
    
    $$\inner{\phi(x)}{y} = \inner{x}{\phi^*(y)}$$
    
    $$\text{данное равенство можно представить в виде: }$$
    
    $$(A_{\phi}^{\mathbb{E}} \;  \smash{\underset{\downarrow}{\xi}})^T \; \Gamma \; \smash{\underset{\downarrow}{\mu}} = \smash{\underset{\downarrow}{\xi}}^T \; \Gamma \; A_{\phi^*}^{\mathbb{E}} \; \smash{\underset{\downarrow}{\mu}}$$
    
    $$\lra  (A_{\phi}^{\mathbb{E}})^T \; \Gamma = \Gamma \; A_{\phi^*}^{\mathbb{E}} \quad \lra \; A_{\phi^*}^{\mathbb{E}} = \Gamma^{-1} \; (A_{\phi}^{\mathbb{E}})^T \; \Gamma$$
    
    $$\text{если ОНБ, то матрица Грамма $\Gamma$ равна единичной}  \lra A_{\phi^*}^{\mathbb{E}} = (A_{\phi}^{\mathbb{E}})^T$$
\end{proof}


\section{Классы линейных операторов в $E$ и $U$}


\begin{shdef}
    \begin{enumerate}
        \item[$\boxed{E:}$]
        \item $\text{Оператор }\phi\text{ называется \textbf{нормальным}, если }\phi \phi^* = \phi^* \phi.$
        
        $\text{Матрица }A\text{ - \textbf{нормальная}, если }A^T A = A A^T$
        
        \item $\text{Оператор }\phi\text{ называется \textbf{самосопряжённым}, если }\phi^* = \phi.$
        
        $\text{Матрица }A\text{ - \textbf{симметричная}, если }A^T = A$
        
        \item $\text{Оператор }\phi\text{ называется \textbf{ортогональным}, если }\phi^* = \phi^{-1}.$
        
        $\text{Матрица }A\text{ - \textbf{ортогональная}, если }A^T = A^{-1}$
    \end{enumerate}
    
    \begin{enumerate}
        \item[$\boxed{U:}$]
        \item $\text{Оператор }\phi\text{ называется \textbf{нормальным}, если }\phi \phi^* = \phi^* \phi.$
        
        $\text{Матрица }A\text{ - \textbf{нормальная}, если }A^* A = A A^*$
        
        \item $\text{Оператор }\phi\text{ называется \textbf{эрмитовым}, если }\phi^* = \phi.$
        
        $\text{Матрица }A\text{ - \textbf{эрмитова}, если }A^* = A$
        
        \item $\text{Оператор }\phi\text{ называется \textbf{унитарным}, если }\phi^* = \phi^{-1}.$
        
        $\text{Матрица }A\text{ - \textbf{унитарная}, если }A^* = A^{-1}$
    \end{enumerate}
\end{shdef}
